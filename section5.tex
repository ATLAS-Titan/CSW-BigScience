%-  LaTeX source file

%-  section5.tex ~~
%
%   This contains what was originally the seventh section of the Google Docs
%   draft of the paper.
%
%                                                   ~~ last updated 20 Sep 2019

The PanDA WMS was originally developed to meet the needs of ATLAS, where it has
been handling production workloads since 2005. PanDA's design was not only
informed by the needs of ATLAS but also optimized to meet them, and similarly,
ATLAS workloads were tailored to run through PanDA. Despite this long history
of co-evolution, nothing about PanDA restricts its use to running physics
simulations on Titan or the WLCG. This section describes some of the
applications of PanDA for managing workloads in other scientific disciplines,
using Titan and other HPC resources.

%%%%%%%%%%%%%%%%%%%%%%%%%%%%%%%%%%%%%%%%%%%%%%%%%%%%%%%%%%%%%%%%%%%%%%%%%%%%%%%%
\subsection{Common Themes for PanDA-Based Workflows}
\label{subsec:common-themes}


Traditionally, the basic pattern for computing in support of physics
experiments is to process input files to produce output files. In this paper,
we have referred to this processing in terms of discrete jobs. These jobs often
require that further input parameters be provided to experiment-specific
software in the form of command-line arguments or additional configuration
files. PanDA can manage a variety of different workloads because it can handle
many different ways of specifying jobs' input and output. For example,
job-specific configuration files can be defined in a PanDA job description
simply as additional input files, and log files created by experiment-specific
software can be defined as additional output files to be saved in the same
tarball that the PanDA pilot creates to save its own log files.


PanDA manages all this processing according to ``transformation scripts'' which
contain full descriptions that provide all necessary information like
configuration and execution parameters for running jobs and saving output. A
PanDA job definition only defines the ``payload'', which is the launching
command for the transformation script. Experiments like ATLAS need to define
complete sets of transformation scripts that cover all possible software usage,
but in many cases, a single transformation script suffices.

PanDA is an attractive solution for new experiments for many reasons. PanDA
provides high-level features for automating the job handling, monitoring, and
logging without removing the ability to control and customize stages in a
workflow. It streamlines the usage of computing resources, especially when
federating resources from multiple facilities, and it insulates users from
details like policies and schedulers that are specific to a local resource.
PanDA provides a diverse set of plugins to support staging data in and out from
remote storages and using different protocols. Another important feature of
PanDA is that is has the ability to integrate closely with the systems at OLCF,
and this makes it attractive both to existing and aspiring OLCF users.


Currently, there are several instances of PanDA Server in use by different
experiments and groups, but in this paper, we have considered three main
instances. The original instance is installed at CERN, and it is used
exclusively for the ATLAS experiment. Another instance is installed at OLCF, as
outlined in Section~\ref{subsec:panda_instance}, and it is dedicated to
supporting projects on Titan, subject to OLCF policies. Finally, the instance
on Amazon's EC2 cloud infrastructure provides access to multiple independent
experiments from different disciplines, and it has the least restrictive
security and usage policies.

Using the OpenShift instance of PanDA outlined in
Section~\ref{subsec:panda_instance}, we created demonstrations for
orchestrating workflows from various scientific fields on OLCF resources. To
isolate the workflows of different groups and experiments, dedicated queues
were defined on each instance of PanDA Server. Separating different groups'
workflows by using different queues provides advantages in customizing
environment variables, system settings, and workflows for different user
groups, and it simplifies job monitoring via the web-based interface. In
collaboration with representatives from other scientific groups, we implemented
transformation scripts as command-line tools that can be addressed by name. A
client tool provided to the users allows them to submit jobs to PanDA Server
with authentication based on grid certificates. Each group's representatives
were authorized to run the pilot launcher daemon with a configurable amount of
concurrency. In this way, the pilots ran and interacted with PBS using the user
and project privileges of the research group, as opposed to running under the
BigPanDA team's account, CSC108.


%%%%%%%%%%%%%%%%%%%%%%%%%%%%%%%%%%%%%%%%%%%%%%%%%%%%%%%%%%%%%%%%%%%%%%%%%%%%%%%%
\subsection{Managing Non-ATLAS Workloads on Titan}
\label{subsec:other-workloads}

The BigPanDA team collaborated with experts from research groups in other
scientific disciplines in order to demonstrate PanDA's abilities to manage
different kinds of workloads on Titan at OLCF. Four representative experiments
are briefly described here, the details for which are summarized in
Table~\ref{tab:beyondhep}.

In the field of genomics, the BigPanDA team collaborated with the Center for
Bioenergy Innovation at ORNL to construct a PanDA-based workflow for genomics
research focused on epistasis. Epistasis is the phenomenon in which the effect
of one gene is dependent on the presence of one or more ``modifier genes''. The
payload used in this workflow is GBOOST \cite{GBOOST}, which is a GPU-based
implementation of the Boolean Operation-based Screening and Testing (BOOST)
software for detecting gene-gene interactions in genome-wide association
studies (GWAS). Each job used only 2 nodes for 30 minutes of wall time to
process 100 MB of input data and produce 300 MB of output data. These jobs were
ideal candidates for using PanDA on Titan because they were small, short, and
GPU-based, which meant that they can take advantage of Titan's hybrid
architecture using backfill opportunities.

In molecular dynamics, the BigPanDA team collaborated with the Chemistry and
Biochemistry Department at the University of Texas at Arlington to construct a
PanDA-based workflow for simulating enzyme catalysis, conformational change,
and ligand binding and release. The payload used in this workflow is CHARMM
(Chemistry at HARvard Macromolecular Mechanics) \cite{Brooks2009CHARMM}, which
is a molecular simulation program capable of using hybrid MPI, OpenMP, and GPU
computing. These jobs ranged in size, but each typically used 124 nodes for
30-90 minutes of wall time. They required just 10 KB of input data files but
produced 2-6 GB of output data. These workloads are interesting because they
can scale to many hundreds of nodes while not requiring large amounts of wall
time, which means they can potentially take great advantage of backfill
opportunities, too.

In particle physics, the BigPanDA team collaborated with the IceCube experiment
\cite{Halzen:2010yj} to construct a PanDA-based workflow for neutrino event
simulation. IceCube is a particle detector at the South Pole that records the
interactions of a nearly massless subatomic particle called the neutrino. The
payload used in this workflow is NUGEN, which is an experiment-specific
software based on ANIS \cite{Gazizov:2004va} that uses GPU computing to
generate samples of atmospheric neutrinos for analysis. These jobs only
required one node at a time for 120 minutes in order to process 500 KB of input
data, but there were many, many jobs to perform as part of their scientific
campaign, and each job produced a volume of output data that could be anywhere
from 10 KB all the way to 4 GB. These workloads were interesting because they
required the use of Singularity containers \cite{} as well as remote stage-in
and stage-out of the data from GridFTP \cite{Allcock:2005:GSG:1105760.1105819}
storage with GSI authentication.

Finally, in neutron science, the BigPanDA team collaborated with the Neutron
Electric Dipole Moment (nEDM) experiment \cite{0954-3899-36-10-104002} to
construct a PanDA-based workflow for processing experimental and observational
data from the Fundamental Neutron Physics Beamline at the Spallation Neutron
Source at ORNL. The goal of the nEDM experiment is to improve the precision of
the measurements of the properties of the neutron in order to search for
violations of fundamental symmetries and to make critical tests of the validity
of the Standard Model of electroweak interactions. The payload used in this
workflow is GEANT \cite{}, which is the same software used by ATLAS to simulate
the passage of particles through matter. These jobs required 200 nodes at a
time, but only for 20 minutes of wall time, in order to process 120 MB of input
data and produce 20 MB of output data. These workloads are particularly
interesting because it is an experimental/observational facility (EOF) which is
local to OLCF resources.


%%%%%%%%%%%%%%%%%%%%%%%%%%%%%%%%%%%%%%%%%%%%%%%%%%%%%%%%%%%%%%%%%%%%%%%%%%%%%%%%
\subsection{Managing Workloads for HPC on Non-OLCF Resources}
\label{subsec:other-hpc-resources}


The BigPanDA team also collaborated on workload management for HPC resources
beyond just Titan at OLCF. Three of these experiments are briefly described
here to represent some of these collaborations, the details for which are
summarized in Table~\ref{tab:beyondhep}.

In the field of astronomy, the BigPanDA team collaborated with the Large
Synoptic Survey Telescope (LSST) project to construct a PanDA-based workflow
for Monte Carlo simulation of image data. Beginning in 2022, LSST will conduct
a 10-year survey of the sky that is expected to deliver 200 PB of data in order
to address some of the most pressing questions about the structure and
evolution of the universe and the objects in it. The payload used in this
workflow is PhoSim \cite{}, a set of fast photon Monte Carlo codes used to
calculate the physics of the atmosphere and a telescope and camera in order to
simulate realistic astronomical images. Each job in the demonstration workflow
used only 2 nodes but required 600 minutes of wall time to process 700 MB of
input data and produce 70 MB of output data. For running LSST simulations with
the PanDA WMS, we have established a distributed testbed infrastructure that
employs the resources of several sites on GridPP \cite{0954-3899-32-1-N01} and
Open Science Grid (OSG) \cite{1742-6596-78-1-012057} as well as the Titan
supercomputer at ORNL. In order to submit jobs to these sites we have used a
PanDA server instance deployed on the Amazon AWS Cloud.

In the field of nuclear physics, and as a part of a SciDAC-4 funded project,
the BigPanDA team collaborated with the Lattice QCD (LQCD)
\cite{Babich:2010:PQL:1884643.1884695} experiment to construct a PanDA-based
workflow to meet the needs of the SciDAC-4 LQCD computational program. LQCD is
a well-established, non-perturbative approach to solving the quantum
chromodynamics theory of quarks and gluons. Current LQCD payloads can be
characterized as massively parallel, occupying thousands of nodes on
leadership-class supercomputers. LQCD payloads have been successfully tested on
Titan as well as on other sites. Production campaigns were executed on the
Institutional Cluster at Brookhaven National Laboratory (BNL) through a
dedicated instance of Harvester \cite{} installed on the site's front node.
Between April and June 2018, PanDA managed workloads that processed 13 TB of
input data and produced 176 GB of output data. LQCD jobs used around 15,000 GPU
hours with an average job duration of approximately 12 hours.

Finally, in the field of neuroscience, the BigPanDA team collaborated with the
Blue Brain Project (BBP) \cite{Markram} of the Ecole Polytechnique Federal de
Lausanne (EPFL) in Switzerland. This proof-of-concept project was aimed at
demonstrating the efficient application of the BigPanDA system to support the
complex scientific workflow of the BBP, which relies on using a mix of
desktops, clusters, and supercomputers to reconstruct and simulate accurate
models of brain tissue. In the first phase of this joint project, we supported
the execution of BBP software on a variety of distributed computing systems
powered by BigPanDA. The targeted systems for demonstration included
cloud-based resources on Amazon Cloud, two clusters using Intel x86 and Nvidia
GPUs located in Geneva and Lugano, one IBM Blue Gene/Q \cite{citeulike:472727}
supercomputer which was also located in Lugano, and of course, Titan at OLCF.


% For tables use
\begin{table}
% table caption is above the table
\caption{Please write your table caption here}
\label{tab:beyondhep}       % Give a unique label
% For LaTeX tables use
\begin{tabular}{llrrrrr}
\hline\noalign{\smallskip}
% ORIGINAL COLUMN TITLES:
%Experiment & Payload/SW & Number of jobs per campaign & Number of nodes per
%job & Walltime (min) & Input data size per job & Output data size per job \\
Experiment & Payload & Jobs & Nodes & Walltime & Input data & Output data \\
\noalign{\smallskip}\hline\noalign{\smallskip}
Genomics           & GBOOST & 10    & 2    & 30 min    & 100 MB & 300 MB \\
IceCube            & NuGen  & 4500K & 1    & 120 min   & 500 KB & 10KB - 4GB \\
LSST/DESC          & Phosim & 20    & 2    & 600 min   & 700 MB & 70 MB \\
LQCD               & QDP++  & 10    & 8000 & 700 min   & 40 GB  & 150 MB \\
Molecular Dynamics & CHARMM & 10    & 124  & 30-90 min & 10 KB  & 2-6 GB \\
nEDM               & GEANT  & 10    & 200  & 20 min    & 120 MB & 20 MB \\
\noalign{\smallskip}\hline
\end{tabular}
\end{table}


%-  vim:set syntax=tex:
