%-  LaTeX source file

%-  beyond_hep.tex ~~
%
%   This contains what was originally the seventh section of the Google Docs
%   draft of the paper.
%
%                                                   ~~ last updated 24 Sep 2018

The objective of each subsection is to:
(i) describe the science; and
(ii) detail what customizations had to be done -- either on PanDA or the Titan
    end to support the science driver.
We will then conclude this section with a summary.

\subsection{PanDA WMS beyond HEP}
\label{subsec:panda_beyond}

Traditionally computing in physics experiments at the basic level is usually
independent processing of the input files to produce the output. This
processing in referring in the paper as a job. Processing algorithm usually
utilizes some experiment-specific software which may require parameterization
and even additional configuration files. In the case if such a configuration
file is specific for each job it can be defined in a job as another input file.
Also experiment software may produce some additional files along with the
primary output and they need to be stored. For instance PanDA pilot itself
produces the tar-archive file containing the logs its own logs and the
experiments software logs. Processing algorithm (referenced as ``transformation
script'') responsible for the correct launching the experiment software and
provide all necessary input information including the configuration and run
parameters. PanDA job definition is only defines the launching command for the
transformation script. This launching command is referring as a payload.

The following components are usually provided and controlled by the experiment groups outside from PanDA core components.
\begin{itemize}
    \item Transformation scripts. User groups should define a complete set of
        the transformations scripts to cover all possible SW usage. In the case
        if the same software is used and only the run parameters, configuration
        and input/output file names are changing, the single transformation
        script should be able to cope this.
    \item Input/output files conventions. The size of the input files often
        adjusted in a way to balance of the total processing walltime and
        flexibility in order to cope the failure risks. There is often case
        that the equal sized input files are required relatively equal
        processing time and produce equal sized output. Also input files are
        often named conventionally and grouped in the datasets by some
        attributes. PanDA job definition allows to provide name for the
        input/output datasets.
\end{itemize}

The real workflow for each scientific group provides a lot of additional
requirements and constraints. A common example is a specific order of the jobs
execution. Also implementation of the dedicated workflows demands an
integration with existing experiment computing infrastructure or even
development an additional components. This includes the issues with data
management, user authentication, monitoring, workflow control and etc.

PanDA system may be the best solution for the new experiments and scientific groups by diversity of provided advantages. The main motivations for users are:
\begin{itemize}
    \item Powerful workload management. Automation of the jobs handling,
        monitoring and logging.
    \item Streamlining the usage of the computing resources. Possibility for
        users to run their jobs on diversity of the computing resources. Local
        resource schedulers, and policies are transparent for the users.
    \item PanDA native data handling. PanDA provides a diverse set of the
        plugins to support data stage-in/-out from the remote storages and
        different data movement tools of different types.
    \item Close integration with OLCF. Being integrated with OLCF PanDA system
        also became attractive for many scientific groups already utilizing
        OLCF resources or those who wish to get use them. 
\end{itemize}

Currently, there are few PanDA instances in use by different experiments and groups. 
In this paper, we have considered three instances. The original instance is installed at CERN, 
and it is used exclusively for the ATLAS experiment. Another instance is installed at OLCF, 
and it is dedicated to supporting projects on Titan, subject to OLCF policies. 
Finally, an instance on Amazon’s EC2 cloud infrastructure provides access to multiple independent 
experiments from different disciplines, and it has the least restrictive security and usage policies.

\subsection{PanDA instance at OLCF}
\label{subsec:panda_instance}

In March 2017, we implemented a new PanDA server instance within OLCF operating
under Red Hat OpenShift Origin [Red Hat OpenShift Origin] - a powerful
container cluster management and orchestration system in order to serve various
experiments at Titan supercomputer. By running on-premise Red Hat OpenShift
built on Kubernetes [Kubernetes], the OLCF provides a container orchestration
service that allows users to schedule and run their HPC middleware service
containers while maintaining a high level of support for many diverse service
workloads. The containers have direct access to all OLCF shared resources such
as parallel filesystems and batch schedulers. With this PanDA instance, we
implemented a set of demonstrations serving diverse scientific workflows
including physics, biology studies of the genes and human brain, and molecular
dynamics studies:
\begin{itemize}
    \item Biology / Genomics. 
In collaboration with Center for Bioenergy Innovation at ORNL the PanDA based
workflow for epistasis researches was established. Epistasis is the phenomenon
where the effect of one gene is dependent on the presence of one or more
``modifier genes'', i.e. the genetic background. GBOOST application [GBOOST], a
GPU-based tool for detecting gene-gene interactions in genome-wide case control
studies, was used for initial tests.
    \item Molecular Dynamics.
In collaboration with Chemistry and Biochemistry department of the University
of Texas Arlington we implemented test to try out PanDA to support the
Molecular Dynamics study ``Simulating Enzyme Catalysis, Conformational Change,
and Ligand Binding/Release''. The CHARMM (Chemistry at HARvard Macromolecular
Mechanics) [CHARMM] a molecular simulation program was chosen as a basic
payload tool. CHARMM design for hybrid MPI/OpenMP/GPU computing.
    \item IceCube. 
Together with  experts from the IceCube experiment we implemented the
demonstrator PanDA system. IceCube [IceCube] is a particle detector at the
South Pole that records the interactions of a nearly massless subatomic
particle called the neutrino. Demonstrator includes the use of NuGen package (a
modified version of ANIS [ANIS] that works with IceCube software) - GPU
application for atmospheric neutrinos are simulations packed in singularity
container and remote stage-in/-out the data from GridFTP [GridFTP] storage with
GSI authentication. 
    \item BlueBrain.
In 2017, a R\&D project was started between BigPanDA and the Blue Brain Project
(BBP) [BBP] of the Ecole Polytechnique Federal de Lausanne (EPFL) located in
Lausanne, Switzerland. This proof of concept project is aimed at demonstrating
the efficient application of the BigPanDA system to support the complex
scientific workflow of the BBP which relies on using a mix of desktop, cluster
and supercomputers to reconstruct and simulate accurate models of brain tissue.
In the first phase of this joint project we supported the execution of BBP
software on a variety of distributed computing systems powered by BigPanDA. The
targeted systems for demonstration included: Intel x86-NVIDIA GPU based BBP
clusters located in Geneva (47 TFlops) and Lugano (81 TFlops), BBP IBM
BlueGene/Q supercomputer [BlueGene](0.78 PFLops and 65 TB of DRAM memory)
located in Lugano, the Titan Supercomputer with peak theoretical performance 27
PFlops operated by the Oak Ridge Leadership Computing Facility (OLCF), and
Cloud based resources such as Amazon Cloud.
    \item LSST.
A goal of LSST (Large Synoptic Survey Telescope) project is to conduct a
10-year survey of the sky that is expected to deliver 200 petabytes of data
after it begins full science operations in 2022. The project  will address some
of the most pressing questions about the structure and evolution of the
universe and the objects in it. It will require a large amount of simulations,
which model the atmosphere, optics and camera to understand the collected data.
For running LSST simulations with the PanDA WMS we have established a
distributed testbed infrastructure that employs the resources of several sites
on GridPP [GridPP] and Open Science Grid (OSG) [OSG] as well as the Titan
supercomputer at ORNL. In order to submit jobs to these sites we have used a
PanDA server instance deployed on the Amazon AWS Cloud.
    \item LQCD.
Lattice QCD (LQCD) [LQCD] is a well-established non-perturbative approach to
solving the quantum chromodynamics theory of quarks and gluons. Current LQCD
payloads can be characterized as massively parallel, occupying thousands of
nodes on leadership-class supercomputers. It is understood that future LQCD
calculations will require exascale computing capacities and workload management
system in order to manage them efficiently.
    \item nEDM.
Precision measurements of the properties of the neutron present an opportunity
to search for violations of fundamental symmetries and to make critical tests
of the validity of the Standard Model of electroweak interactions. These
experiments have been pursued [neutron] with great energy and interest since
the discovery of neutron in 1932. The goal of the nEDM [nEDM] experiment at the
Fundamental Neutron Physics Beamline at the Spallation Neutron Source (Oak
Ridge National Laboratory) is to further improve the precision of this
measurement by another factor of 100.
\end{itemize}

To isolate the workflows of different groups and experiments, dedicated queues
were defined at the PanDA server. Presumably in  next steps we will provide the
security mechanisms that will provide the access to each queue for job
submission and dispatching only for authorised users. Also, the PanDA server
provides the tools to customise environment variables, system settings and
workflow algorithms for different user groups. Also this split of the different
groups workflows on the level of PanDA queues simplifies jobs monitoring via
the web based PanDA tool. 

In collaboration with the dedicated scientific groups representatives, we
implemented the ``transformation'' scripts containing complete definition of
the processing actions (set of specific software and general system commands)
are has to be applied to the input data to produce the output. The
transformation script then can be addressed by its name. Client tool provided
to the users allows to submit jobs to the PanDA server with authentication
based on grid certificates. 

Responsible group representative also authorized to run pilots launcher daemon.
Daemon launches the pilots. Number of parallel running pilots can be
configured. Pilots are running and interacts with the PBS under user account
and with Titan group privileges of the responsible representative.

The most important parameters of conducted tests are presented in the table

% For tables use
\begin{table}
% table caption is above the table
\caption{Please write your table caption here}
\label{tab:1}       % Give a unique label
% For LaTeX tables use
\begin{tabular}{llrrrrr}
\hline\noalign{\smallskip}
% ORIGINAL COLUMN TITLES:
%Experiment & Payload/SW & Number of jobs per campaign & Number of nodes per
%job & Walltime (min) & Input data size per job & Output data size per job \\
Experiment & Payload & Jobs & Nodes & Walltime & Input data & Output data \\
\noalign{\smallskip}\hline\noalign{\smallskip}
Genomics           & GBOOST & 10    & 2    & 30 min    & 100 MB & 300 MB \\
Molecular Dynamics & CHARMM & 10    & 124  & 30-90 min & 10 KB  & 2-6 GB \\
IceCube            & NuGen  & 4500K & 1    & 120 min   & 500 KB & 10KB - 4GB \\
LSST/DESC          & Phosim & 20    & 2    & 600 min   & 700 MB & 70 MB \\
LQCD               & QDP++  & 10    & 8000 & 700 min   & 40 GB  & 150 MB \\
nEDM               & GEANT  & 10    & 200  & 20 min    & 120 MB & 20 MB \\
\noalign{\smallskip}\hline
\end{tabular}
\end{table}


\subsection{Summary}
\label{subsec:summary}

The overview of the successfully implemented demonstrations of diverse
workflows implementation via PanDA shows that PanDA model can cope the
challenges of the different experiments and user groups and also provide
possibility for extensions beyond the core components set.  The proof of
concept was received from all considered experiments representatives and
results that PanDA is considered as a possible solution. Preproduction
utilization of PanDA is now under investigation with BlueBrain, IceCube, LSST,
nEDM experiments, LQCD uses PanDA for Production.

%-  vim:set syntax=tex:
