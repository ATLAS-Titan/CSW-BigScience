%-  LaTeX source file

%-  main.tex ~~
%
%   This is the "main" document file, which means it is the one that will
%   \include all the other source files.
%
%                                                   ~~ last updated 25 Sep 2019

%%%%%%%%%%%%%%%%%%%%%%%%%%%%%%%%%%%%%%%%%%%%%%%%%%%%%%%%%%%%%%%%%%%
%
% This is a general template file for the LaTeX package SVJour3
% for Springer journals.          Springer Heidelberg 2010/09/16
%
% Copy it to a new file with a new name and use it as the basis
% for your article. Delete % signs as needed.
%
% This template includes a few options for different layouts and
% content for various journals. Please consult a previous issue of
% your journal as needed.
%
%%%%%%%%%%%%%%%%%%%%%%%%%%%%%%%%%%%%%%%%%%%%%%%%%%%%%%%%%%%%%%%%%%%
%
\RequirePackage{fix-cm}
%
%\documentclass{svjour3}                     % onecolumn (standard format)
%\documentclass[smallcondensed]{svjour3}     % onecolumn (ditto)
\documentclass[smallextended]{svjour3}      % onecolumn (second format)
%\documentclass[draft,smallextended]{svjour3} % second format in draft mode
%\documentclass[twocolumn]{svjour3}          % twocolumn

%-  LaTeX source code

%-  include.tex ~~
%
%   This includes material from the original template provided by the journal,
%   as well as modifications. This file exists in order to match the project
%   structure of our group's eScience 2017 paper.
%
%                                                       ~~ (c) SRW, 22 Sep 2018
%                                                   ~~ last updated 22 Sep 2018

% <stuff-from-original-template>
%
\smartqed  % flush right qed marks, e.g. at end of proof
%
\usepackage{graphicx}
%
% \usepackage{mathptmx}      % use Times fonts if available on your TeX system
%
% insert here the call for the packages your document requires
%\usepackage{latexsym}
% etc.
%
% please place your own definitions here and don't use \def but
% \newcommand{}{}
%
% Insert the name of "your journal" with
% \journalname{myjournal}
%
% </stuff-from-original-template>

\usepackage{amsmath}

%-  vim:set syntax=tex:


\begin{document}

\title{Improving Resource Utilization by High-throughput Workload Management
at Leadership Class Facilities%
% \thanks{\textcolor{blue}{Grants or other notes about the article that should
% go on the front page should be placed here. General acknowledgments should be
% placed at the end of the article.}}
}
%\subtitle{\textcolor{blue}{Do you have a subtitle?\\ If so, write it here}}

%\titlerunning{Short form of title}        % if too long for running head

\author{%
    Sean R.\ Wilkinson~$^1$ \and
    Danila Oleynik~$^2$ \and
    Ruslan Mashinistov~$^3$ \and
    Andre Merzky~$^4$ \and
    Sergey Panitkin~$^3$ \and
    Pavlo Svirin~$^5$ \and
    Matteo Turilli~$^4$ \and
    Kaushik De~$^5$ \and
    Alexei Klimentov~$^3$ \and
    Shantenu Jha~$^{3, 4}$ \and
    Jack C.\ Wells~$^1$
}

\authorrunning{Sean R.\ Wilkinson et al.} % if too long for running head

\institute{%
    $^1$~Oak Ridge National Laboratory,
        Oak Ridge, Tennessee, USA \\
    $^2$~Joint Institute for Nuclear Research,
        Dubna, Moscow Oblast, Russia \\
    $^3$~Brookhaven National Laboratory,
        Upton, New York, USA \\
    $^4$~Rutgers, the State University of New Jersey,
        New Brunswick, New Jersey, USA \\
    $^5$~University of Texas at Arlington,
        Arlington, Texas, USA
%    S. Wilkinson \at Oak Ridge National Laboratory \\
%        \email{wilkinsonsr@ornl.gov}
%              first address \\
%              Tel.: +123-45-678910\\
%              Fax: +123-45-678910\\
%              \email{fauthor@example.com}           %  \\
%%             \emph{Present address:} of F. Author  %  if needed
%           \and
%           S. Author \at
%              second address
}

\date{Received: date / Accepted: date}
% The correct dates will be entered by the editor

\maketitle

% ---------------------------------------------------------------------------
% ABSTRACT
% ---------------------------------------------------------------------------

\begin{abstract}
Large experiments and big science projects have historically used distributed
resources to meet their computing requirements. On the other hand, leadership
class machines have traditionally been utilized for workloads with very
distinct properties, namely, large MPI jobs. Given this gap, two interesting
questions are: can leadership computing facilities support workloads that
historically used distributed high-throughput resources? Can workload
management systems that were designed and implemented to support distributed
high throughput computing (HTC) support the efficient execution of workloads
on leadership class platforms? This paper documents our experience in enabling
PanDA --- Production and Distributed Analysis (PanDA) originally developed for
the ATLAS experiment at the Large Hadron Collider (LHC) at CERN, as a
production workload management on Titan --- a leadership class machine that
operated from 2012 until 2019. PanDA was modified to interact with the Moab
resource manager in two distinct operational modes: a ``batch queue mode'' that
used traditional allocations, and a ``backfill mode'' that opportunistically
consumed otherwise unutilized resources. For the traditional mode, new
techniques were implemented to shape large jobs for consuming allocations on a
leadership-class machine. In backfill mode, workloads was executed as backfill
to high priority leadership-class jobs but the stated goal of utilizing
resources without impacting Titan's quality of service as experienced by other
jobs. We propose several metrics by which to measure the impact; our analysis
suggests that with high probability and degree of confidence, the ATLAS
project was able to utilize 420 million core hours in backfill mode without a
discernible impact on other jobs and also increased the overall utilization of
Titan. Finally, we describe the use of PanDA for experiments in other
scientific fields such as astronomy, genomics, and neuroscience, using both
OLCF resources and other HPC resources.
\keywords{ATLAS \and HTC \and HPC \and PanDA \and OLCF \and Titan \and WMS}% \PACS{PACS code1 \and PACS code2 \and more}
% \subclass{MSC code1 \and MSC code2 \and more}
\end{abstract}

% The Production and Distributed Analysis (PanDA) software was originally
% developed as a High Throughput Computing (HTC) workload management system (WMS)
% on distributed grid resources for the ATLAS experiment at the Large Hadron
% Collider (LHC) at CERN. The BigPanDA team extended PanDA to access High
% Performance Computing (HPC) resources, allowing leadership-class supercomputers
% like Titan at Oak Ridge Leadership Computing Facility (OLCF) to become grid
% sites for the Worldwide LHC Compute Grid (WLCG). ATLAS consumed more than 620
% million Titan core hours from 2016-2018 as part of its production workflow for
% simulations by extending PanDA to interact with the Moab resource manager in
% two distinct operational modes: a ``batch queue mode'' that used traditional
% allocations to consume 200 million core hours, and a ``backfill mode'' that
% opportunistically consumed unutilized resources totaling more than 420 million
% core hours. For the traditional mode, new techniques were implemented to shape
% large jobs for consuming allocations on a leadership-class machine. In backfill
% mode, work was streamed steadily to Titan to backfill high priority
% leadership-class jobs, with the goal of consuming unutilized resources that
% would have otherwise remained unutilized, without affecting other projects'
% quality of service. This goal was evaluated with several studies that are
% included here, and the studies concluded that the goal was achieved. Finally,
% this report describes the use of PanDA for experiments in other scientific
% fields. Multiple demonstration workflows are described for fields such as
% astronomy, genomics, and neuroscience, using both OLCF resources and other HPC
% resources. 

% ---------------------------------------------------------------------------
% I - INTRODUCTION
% ---------------------------------------------------------------------------

\section{Introduction}
\label{sec:introduction}
%-  LaTeX source file

%-  section1.tex ~~
%                                                   ~~ last updated 24 Sep 2018

The Production and Distributed Analysis (PanDA) software was originally
developed as a High Throughput Computing (HTC) workload management system (WMS)
on distributed grid resources for the ATLAS experiment at the Large Hadron
Collider (LHC) at CERN. Traditionally, the ATLAS experiment at LHC has utilized
distributed resources as provided by the Worldwide LHC Compute Grid (WLCG) to
support data distribution and enable the simulation of events. The WLCG is the
world's largest computing grid, boasting more than 1 million computer cores and
1 exabyte of storage combined from more than 170 individual computing centers
in 42 countries. ATLAS experiment uses this geographically distributed grid to
process, simulate, and analyze its data, which total more than 350 petabytes.
After the early success in discovering a new particle consistent with the
long-awaited Higgs boson, ATLAS continued taking the precision measurements
necessary for further discoveries during Run 2, which came to an end in
December 2018. Especially in light of the future Run 3 as well as the High
Luminosity LHC (HL-LHC) project, it is obvious that the need for simulation and
analysis will overwhelm the expected capacity of WLCG computing facilities
unless the range and precision of physics studies are curtailed.

Over the past few years, the ATLAS experiment has been investigating the
implications of using high-performance computers -- such as those found at Oak
Ridge Leadership Computing Facility (OLCF) and other United States Department
of Energy (DOE) computing facilities -- to augment WLCG computing facilities by
integrating their High Performance Computing (HPC) resources. This steady
transition is a consequence of application requirements such as greater than
expected data production, as well as technology trends and software complexity.

To this end, the BigPanDA team extended PanDA to access HPC resources, allowing
leadership-class supercomputers like Titan at OLCF to become grid sites for the
WLCG. HPC resources are not necessarily tuned for processing data on the scale
of the ATLAS experiment, but they are incredibly well-suited for simulation
work. Taking advantage of the strengths of HPC resources like Titan allowed
ATLAS to consume more than 620 million Titan core hours during the three years
from 2016-2018 as part of its production workflow for simulations. Not only did
this represent a large proportion of ATLAS's resource consumption for
simulations, but it also led to the incorporation of other DOE HPC resources as
WLCG grid sites, including Theta at the Argonne Leadership Computing Facility
(ALCF) and Cori at the National Energy Research Scientific Computing Center
(NERSC).

Here, we describe the architectural, algorithmic, and software changes which
were addressed in order to prepare the PanDA WMS for the exascale. We will
focus on our experiences in adapting PanDA to work with Titan at the OLCF. Even
though Titan was decommissioned in August 2019, these experiences are directly
applicable to Summit, OLCF's pre-exascale successor to Titan. In particular, we
will discuss the extensions to PanDA that allowed it to interact with Titan's
Moab resource manager in two distinct operational modes: a ``batch queue mode''
that used traditional allocations to consume 200 million core hours, and a
``backfill mode'' that opportunistically consumed unutilized resources totaling
more than 420 million core hours. For the traditional mode, we will describe
new techniques that were implemented to shape large jobs for consuming DOE
Advanced Scientific Computing Research (ASCR) Leadership Computing Challenge
(ALCC) allocations on Titan.

We also describe the use of a ``backfill mode'' for PanDA in which work was
streamed steadily to Titan and reshaped and submitted dynamically based on
available unutilized resources. The BigPanDA team took advantage of the
backfill scheduling feature of the Moab resource manager with the goal of
consuming unutilized resources that would have otherwise remained unutilized,
without affecting other projects' quality of service. We assess the
performance of PanDA with respect to this goal with several studies that are
included here.

Finally, this report describes the use of PanDA for experiments in other
scientific fields. Multiple demonstration workflows are described for fields
such as astronomy, genomics, and neuroscience, using both OLCF resources and
other HPC resources.

%-  vim:set syntax=tex:


% ---------------------------------------------------------------------------
% II - PanDA Workload Management System: Software System Overview
% ---------------------------------------------------------------------------

\section{PanDA Workload Management System: Software System Overview}
\label{sec:overview}
\input{section2}

% ---------------------------------------------------------------------------
% III - Deploying PanDA Workload Management System on Titan
% ---------------------------------------------------------------------------

\section{Deploying PanDA Workload Management System on Titan}
\label{sec:deploying}
%-  LaTeX source file

%-  section3.tex ~~
%
%   This is the third section of the paper.
%
%                                                   ~~ last updated 22 Sep 2019

\jhanote{Need to set the tense correctly: refer to Titan in the past}

\jhanote{consistency between Moab and MOAB}

\jhanote{backfill versus backfill mode vs ``as a backfill job''}

%%%%%%%%%%%%%%%%%%%%%%

In 2013, the BigPanDA team began work on behalf of the ATLAS Collaboration to
incorporate the Titan supercomputer at the Oak Ridge Leadership Computing
Facility (OLCF) as a grid site for the Worldwide LHC Compute Grid, and in 2019,
Titan was decommissioned. This section describes the deployment of the PanDA
WMS on Titan as well as the relevant policies at OLCF and PanDA's integration
with the Moab workload manager \cite{moab}. PanDA operated on Titan in two very
different modes of operation, colloquially termed ``batch queue mode'' and
``backfill mode''.  In ``batch queue mode'', PanDA interacted with Titan's Moab
scheduler in a static, non-adaptive manner to execute the workload. In
``backfill mode'', PanDA dynamically shaped the size of the workload deployed
on Titan to consume resources opportunistically that would otherwise have gone
unused.

%%%%%%%%%%%%%%%%%%%%%%%%%%%%%%%%%%%%%%%%%%%%%%%%%%%%%%%%%%%%%%%%%%%%%%%%%%%%%%%%
\subsection{OLCF Policies}
\label{subsec:olcf-policies}

Oak Ridge Leadership Computing Facility (OLCF) is a United States Department of
Energy (DOE) ``leadership computing facility'' with a mission to enable
applications of size and complexity that cannot be readily performed at smaller
facilities. The OLCF has a mandate that a large portion of its flagship
machines' usage must come from large, leadership-class jobs, which are also
known as ``capability jobs''. Thus, the OLCF prioritizes the scheduling of
capability jobs.

To ensure that the OLCF user programs achieved this mission with Titan, OLCF
policies strongly encouraged users to run jobs that were as large as their
codes would allow. There were three queues on Titan, which were ``batch'',
``debug'', and ``killable''. OLCF used batch queue policy on the Titan systems
to support the delivery of large capability-class jobs~\cite{titan_sched}. OLCF
deployed Adaptive Computing's Moab resource manager, which supported features
that allowed it to integrate directly with Cray's Application Level Placement
Scheduler (ALPS), a lower-level resource manager unique to Cray HPC
clusters~\cite{osti_1086656}. Moab scheduled jobs in the batch queue in
priority order, and the highest priority jobs were executed depending on the
availability of the required resources. The OLCF therefore implemented queue
policies which awarded the highest priority to the largest capability jobs,
rather than just the oldest jobs in the batch queue.

The highest priority jobs were the ones next in line to run, unless the job did
not fit, which could happen, for example, when the requested resources were not
available. In such a case, a resource reservation was made for the job in the
future when availability could be assured; those nodes were exclusively
reserved for that job. When the job finished, the reservation was destroyed,
which released those nodes so that they were available for the next job.
Reservations were simply the mechanism by which a job received exclusive access
to the resources necessary to run the job. However, if policy desired that a
priority reservation be made for more than one job, then a system administrator
could specify the creation of reservations for the top N priority jobs in the
queue by increasing the keyword RESERVATIONDEPTH to be greater than one. The
priority reservation(s) would be re-evaluated (and destroyed or re-created)
during every scheduling iteration in order to take advantage of updated
information. 

Of course, reservations seldom filled the nodes on Titan exactly, and Moab
would schedule smaller jobs to run in the vacancies. For example, if a large
capability job was due to start in two hours, Moab would work backwards to fill
in, or ``backfill'', the vacant nodes with the highest priority jobs in the
queue capable of finishing in less than two hours. The situation in which there
are vacant nodes for some amount of time was called colloquially ``backfill
opportunity'' by the BigPanDA team.

Thus, after creating reservations for the top priority jobs, Moab would switch
to ``backfill mode'' and continue down the job queue until it found a job that
would be able to start and would not disturb the existing priority
reservations, as specified by the value of RESERVATIONDEPTH. As time continued
and the scheduling algorithm continued to iterate, Moab would evaluate the
queue to find the highest priority jobs. If the highest priority job found
would not fit within the available resources, its reservation was updated but
left where it was. At this point, Moab would begin to try to backfill vacancies
by searching for a job in the queue that would be able to start and complete
without disturbing the priority reservations. If such jobs were started, they
were said to ``run within backfill''. If no such backfill jobs were present in
the queue, then available compute resources remained unutilized. 

It is important to note, however, that there was no dedicated ``backfill
queue'' for Titan; instead, smaller jobs from each queue were scheduled into
spaces that could not be used by larger jobs. There were three queues on Titan,
which were ``batch'', ``debug'', and ``killable''. The batch queue was the
default queue for submitted jobs, and this paper is concerned only with the
batch queue.

Jobs submitted to the batch queue were grouped into five ``bins'' according to
the number of requested nodes, and each bin had a maximum wall time. The
definitions and rules for each bin are shown in Table~\ref{tab:olcf-bins}. Jobs
that requested fewer nodes had correspondingly lesser maximum wall times. Nodes
were assigned exclusively to one job at a time. Because Titan was a leadership
class machine and priority was a function of wait time, the batch scheduler
awarded aging boosts to jobs in bins 1 and 2 in order to prioritize larger jobs
over smaller ones. Once jobs in the batch queue begin to run, however, they
were not killed when new jobs arrived, regardless of their priority. Sometimes,
jobs small enough to use currently idle resources on Titan were scheduled to
run immediately. Finally, ``Titan core hours'' were the billable units used at
OLCF; they converted at a rate of 30 Titan core hours per 1 node hour.

%%%
% OLCF BINS TABLE
%%%
% For tables use
\begin{table}
% table caption is above the table
\caption{OLCF policies sort jobs into numbered bins based on the requested
number of nodes, and each bin has its own set of constraints.}
\label{tab:olcf-bins}       % Give a unique label
% For LaTeX tables use
\begin{tabular}{crrr}
\hline\noalign{\smallskip}
Bin & Requested Nodes   & Maximum Wall Time &   Aging Boost \\
\noalign{\smallskip}\hline\noalign{\smallskip}
1   &   11,250 - 18,688 &   24 hours        &   15 days     \\
2   &    3,750 - 11,249 &   24 hours        &    5 days     \\
3   &       313 - 3,749 &   12 hours        &         0     \\
4   &         126 - 312 &    6 hours        &         0     \\
5   &           1 - 125 &    2 hours        &         0     \\
\noalign{\smallskip}\hline
\end{tabular}
\end{table}

%%%%%%%%%%%%%%%%%%%%%%%%%%%%%%%%%%%%%%%%%%%%%%%%%%%%%%%%%%%%%%%%%%%%%%%%%%%%%%%%
\subsection{PanDA Integration at OLCF}
\label{subsec:panda-at-olcf}

In 2013, the BigPanDA team began working on behalf of the ATLAS Collaboration
to incorporate the Titan supercomputer at OLCF as a grid site for the Worldwide
LHC Compute Grid. The team operated under several different project
identifiers, including CSC108, HEP110, and HEP113. The HEP110 and HEP113
projects represented traditional ASCR Leadership Computing Challenge (ALCC)
allocations, but the CSC108 project was a Director's Discretionary (DD) project
which operated exclusively in what the team colloquially referred to as
``backfill mode'', as outlined in Section~\ref{subsec:olcf-policies}.

PanDA is a pilot-based WMS. In a distributed computing setting, pilot jobs are
submitted to batch queues on compute sites, and then they wait for resources to
become available. When a pilot job starts on a worker node, it contacts the
PanDA Server to retrieve an actual payload, and then, after necessary
preparations, it executes the payload as a sub-process. The PanDA pilot is also
responsible for a job's data management on a worker node and is capable of
performing data stage-in and stage-out operations.

Taking advantage of PanDA's modular and extensible design, the BigPanDA team
enhanced the pilot code and logic with tools and methods relevant for work on
HPCs. The pilots ran on Titan's data transfer nodes (DTNs), which allowed them
to communicate with the ATLAS PanDA Server, since the DTNs had good (10 GB/s)
connectivity to the Internet. The DTNs and the worker nodes on Titan used a
shared filesystem which enabled the pilot to stage-in the input files required
by the payload and to stage-out the output files produced at the end of the
job. In other words, the pilot acted as a site edge service for Titan. Pilots
were launched by a daemon-like script which ran in user space.

The ATLAS Tier 1 computing center at Brookhaven National Laboratory was used
for data transfer to and from Titan, but in principle any ATLAS site could have
been chosen. Figure~\ref{fig:implementation} shows a schematic view of PanDA's
interface with Titan. The pilots submitted ATLAS payloads to the worker nodes
using the local batch system via the Simple API for Grid Applications (SAGA)
interface \cite{saga_cmst}. SAGA was also used for monitoring and
management of PanDA jobs running on Titan's worker nodes.

One interesting feature of the deployment was its ability to collect and use
information about Titan's status, such as its free worker nodes, in real time.
The pilot could query the Moab scheduler about currently unused nodes on Titan
by using the ``showbf'' command-line tool, and the pilot could check to see if
the free resources' availability and size represented a suitable ``backfill
opportunity'' for PanDA jobs. The pilot transmitted this information to the
PanDA Server, which responded by sending the pilot a list of jobs intended for
submission to Titan. Then, based on the job information, the pilot transferred
the necessary input data from the ATLAS Grid, and once all the necessary data
was transferred, the pilot submitted jobs to Titan using an MPI wrapper. 

The MPI wrappers were Python scripts that were typically workload-specific,
since they were responsible for setup of the workload environment, organization
of per-rank worker directories, rank-specific data management, optional input
parameter modification, and cleanup on exit. When activated on worker nodes,
each copy of the wrapper script would, after completing the necessary
preparations, start the actual payload as a sub-process and wait until its
completion. This approach allowed for flexible execution of a wide spectrum of
grid-centric workloads on parallel computational platforms such as Titan
\cite{htchpc2017converging}.

Because ATLAS detector simulations were executed on Titan as discrete jobs
submitted via MPI wrappers, parallel performance could scale nearly linearly,
potentially limited only by shared filesystem performance. Up to 20 pilots were
deployed at a time, distributed evenly over 4 DTNs. Each pilot controlled from
15 to 350 ATLAS simulation ranks per submission. This configuration was able to
utilize up to 112,000 cores simultaneously on Titan.

Figure~\ref{fig:monthly-consumption} shows Titan core hours consumed per month
by the ATLAS Geant4 simulations from January 2016 through December 2018. During
this time, CSC108 always ran in pure backfill mode with a custom priority that
was guaranteed to make its jobs the lowest priority on Titan, and the project
also had no actual allocation. Despite these obstacles, CSC108 still consumed
more than 400 million Titan core hours during that three year time period,
peaking at more than 24 million Titan core hours during the month of October
2018. The drop in consumption that occurred in
Figure~\ref{fig:monthly-consumption} during the months of July, August, and
September 2018 was due to the end of a major ATLAS simulation campaign in June
2018; there were simply no simulation jobs to run.

%%%%%%%%%%%%%%%%%%%%%%%%%%%%%%%%%%%%%%%%%%%%%%%%%%%%%%%%%%%%%%%%%%%%%%%%%%%%%%%%
\subsection{PanDA Server at OLCF}
\label{subsec:panda_instance}

The PanDA Server used to manage ATLAS production workloads was the dedicated
instance at CERN in Geneva, Switzerland. Thus, it was necessary to deploy
another instance of the PanDA Server elsewhere in order to manage non-ATLAS
workloads on Titan. To this end, in March 2017, the BigPanDA team implemented a
new PanDA Server instance within the OLCF by using Red Hat OpenShift Origin
\cite{RH_OpenShift}, a powerful container cluster management and orchestration
system.

By running PanDA Server on OLCF premises with Red Hat OpenShift built on
Kubernetes \cite{Kubernetes}, the OLCF provided a container orchestration
service that allowed the BigPanDA team as users to schedule and run HPC
middleware service containers while maintaining a high level of support for
many diverse service workloads. The containers had direct access to all shared
resources at the OLCF, including parallel filesystems and batch schedulers.
This PanDA Server instance was used to implement the demonstrations for
non-ATLAS workloads that are detailed in
Section~\ref{sec:beyond-atlas-and-olcf}.


% For two-column wide figures use
\begin{figure*}
% Use the relevant command to insert your figure file.
% For example, with the graphicx package use
  \includegraphics[width=0.75\textwidth]{images/Figure_5.png}
% figure caption is below the figure
\caption{Schematic view of PanDA WMS integration with Titan supercomputer at OLCF}
\label{fig:implementation}
\end{figure*}


% For two-column wide figures use
\begin{figure*}
% Use the relevant command to insert your figure file.
% For example, with the graphicx package use
  \includegraphics[width=0.75\textwidth]{images/monthly-consumption.png}
% figure caption is below the figure
\caption{This figure shows the monthly consumption of resources on Titan by the
two methods used by PanDA.}
\label{fig:monthly-consumption}
\end{figure*}

%-  vim:set syntax=tex:


% ---------------------------------------------------------------------------
% IV - Performance Characterization on Titan
% ---------------------------------------------------------------------------

\section{Impact Assessment of PanDA on Titan}
\label{sec:impact-assessment}
\jhanote{Performance of what? We should rename this section to some thing less
ambigious}
\seannote{Pretty much every section and subsection title can be tweaked, even
if only to make sure they all match the same parts of speech. I renamed it, but
obviously we can re-rename it.}

%-  LaTeX source file

%-  section4.tex ~~
%
%   This is the fourth section of the paper.
%
%                                                   ~~ last updated 22 Sep 2019

%%%%%%%%%%%%%%%%%%%%%%%%%%%%%%%%%%%%%%%%%%%%%%%%%%%%%%%%%%%%%%%%%%%%%%%%%%%%%%%

As outlined in Section~\ref{subsec:panda-at-olcf}, the BigPanDA team operated
under several OLCF project identifiers, including CSC108, a DD project which
sought to take advantage of ``backfill mode'' in order to consume unutilized
resources on Titan which would otherwise have gone to waste, without affecting
other projects' quality of service. The focus of this section is to assess the
degree to which CSC108 accomplished that goal and especially to analyze the
impact the project made on Titan.

\jhanote{there is a difference between ``idle'' and ``unutilized''. Need more
rigour in the definition of the stated goals}
\seannote{I admit I don't know the difference. What I generally intend is to
refer to machine which are available to perform work but aren't working.}
\jhanote{Idle is a state when not running. Underutilized is a ``response'' measured against a notion of adequate or desired utilization.}

% \jhanote{slightly misleading to claim there are three queues then to say
% immediately there is no backfill queue. Haven't we already discussed this in
% Section 3?}
% \seannote{It was confusing as written, so I moved the discussion of queues into
% Section 3.2. There was no backfill queue -- backfill was an optional feature
% that could be added to existing queues, and we played fast and loose in the
% past with how we referred to a ``queue'' that didn't actually exist.}

As a convention, we will sometimes make use of a common scheduling metaphor in
which jobs waiting to run on a computer are represented by rocks that are being
used to fill a jar. Capability class jobs on Titan are the largest rocks, and
the scheduler typically fills the remaining space around the largest rocks
with smaller rocks. CSC108 attempted to fill whatever space remains, thanks
to its having been given the lowest possible priority.

%%%%%%%%%%%%%%%%%%%%%%%%%%%%%%%%%%%%%%%%%%%%%%%%%%%%%%%%%%%%%%%%%%%%%%%%%%%%%%%%
\subsection{Rescheduling study}
\label{subsec:rescheduling-study}

% \jhanote{Please make the formulation of the statement more precise}
% \seannote{I'm not sure I made it any better, but I tried.} 

\jhanote{the following is a very strong statement. Discuss}

Recall that the goal of CSC108 was to consume unutilized resources on Titan
that would otherwise have remained unutilized, without decreasing the quality
of service experienced by other projects on Titan.  As shown in
Figure~\ref{fig:monthly-consumption}, CSC108 consumed hundreds of millions of
core hours on Titan. These resources are guaranteed to have been unutilized at
the time of that CSC108's jobs were scheduled because during that time period,
there was no pre-emption in the batch queue on Titan. Moreover, even if Moab
had ended running jobs in order to run other jobs with higher priorities
instead, CSC108's jobs would never have been used to replace existing jobs
because of the custom policies in place that guaranteed CSC108 would always
have the lowest priority on Titan. Therefore, CSC108 consumed millions of core
hours on Titan, all of which were guaranteed to have been unutilized at the
time that CSC108's jobs were scheduled.

It is a more difficult problem, however, to prove that these unutilized
resources on Titan would have remained unutilized and therefore ``gone to
waste''. For example, although the resources were unutilized at the time that
CSC108's jobs were scheduled, if other projects submitted jobs to Titan before
CSC108's jobs completed, then CSC108's jobs could be said to have ``blocked''
other jobs from running. This competition for resources would immediately imply
that, although some of the resources consumed by CSC108 would have gone to
waste, the rest of those resources would not have been wasted.

The simple bar plot shown in Figure~\ref{fig:jacks-slide} seems to show, at
first glance, that utilization by CSC108 has increased at the same time that
utilization by other projects has decreased, which is suggestive of competition
for resources. As a simple way to investigate this competition for resources,
we conducted an experiment in which three years' worth of job traces from Titan
were rescheduled, with and without CSC108's jobs, so that we could estimate
the impact of CSC108's jobs on wait times, throughput, and utilization for
Titan.

\jhanote{Hypothesis as currently formulated needs greater clarity: ``Has no
effect on Titan'' is too casual and clearly wrong, because it does increase
the utilization!} \seannote{I changed it to say ``null hypothesis''. It is
definitely a null hypothesis. The problem then, I suppose, is the implication
then that I will be seeking a p-value to test it.}

% \jhanote{Difficult to read /understand algorithm as is presented.}
% \seannote{I have updated to explain with more words and simpler pseudocode.}
% For two-column wide figures use
\begin{figure*}
% Use the relevant command to insert your figure file.
% For example, with the graphicx package use
  \includegraphics[width=0.75\textwidth]{images/barplot-jacks-slide.png}
% figure caption is below the figure
\caption{This bar plot visualizes the yearly utilization of Titan core hours,
separated into two categories: CSC108's utilization of Titan core hours through
backfill opportunities (shown in red), and all other utilization of core hours
on Titan (shown in blue).}
\label{fig:jacks-slide}
\end{figure*}

What we mean by ``compressing'' job traces is rescheduling the jobs that ran
on Titan while preserving the original execution order, under the assumption of
100\% node availability (all nodes up and running all the time). 

In order to reschedule the jobs, we made a simplifying assumption of 100\% node
availability, meaning that we assumed all of Titan's nodes were up and running
all the time. Original job execution order was also preserved. We rescheduled
the jobs once with CSC108's jobs included and then once with CSC108's jobs
removed, using the algorithm shown in Listing~\ref{lst:rescheduling.py} as
pseudo-code in the form of Python.

The idea here is akin to Archimedes's famous insight to measure the volume of
an irregularly shaped solid by measuring the difference in the volume of a
fluid before and after submerging the solid. In this case, however, we are
trying to evaluate the fluid, because we are most interested in the fluid's
ability to fill the irregular ``spaces'' between the other projects' jobs. The
difference in the completion times for the last scheduled job, with and without
CSC108, provide insight about the ability of CSC108's jobs to fit precisely
in the spaces between other jobs. This rescheduling technique also allows
CSC108's jobs' effects on throughput and utilization to be estimated in a
simple way, by simply computing them with and without CSC108's jobs included
with the other job traces. Throughput was defined as jobs completed per day,
and utilization was defined as the percentage of available core hours which
were consumed.

The data used for this experiment are historical traces for all jobs on Titan
during the years 2016, 2017, and 2018. These data were provided in anonymized
form by OLCF so that job, project, and user identifiers were included only for
the three PanDA projects on Titan. Otherwise, the job traces consist only of
job submission time, start time, completion time, the number of requested
nodes, and the amount of wall time requested. The experimental programs were
written in Python using the well-known Matplotlib, NumPy, and SciPy libraries
via Anaconda. Data were originally provided as text files but were imported
into a SQLite database.

\lstinputlisting[language=Python,
    label={lst:rescheduling.py},
    caption={Job traces are assumed to be sorted in order of original start
                time.}]{rescheduling.py}

The results, which are summarized in Table~\ref{tab:rescheduling-results}, show
that including CSC108's jobs when rescheduling results in the simulation taking
13.3 additional days of simulation time, which represents a 1.30\% increase in
the total time to completion. Throughput increases by 190.26 jobs per day,
which represents a 14.36\% increase, and utilization increases from 92.36\% to
94.15\%, which is a 1.94\% increase when CSC108's jobs are included versus when
they are omitted.

If CSC108's jobs fit perfectly within the spaces between the other projects'
jobs, then we would expect there to be no difference between the simulation's
times to completion with and without CSC108's jobs. The simulation observed a
1.30\% increase in time to completion, however, which indicates that CSC108's
jobs did not fit perfectly within the spaces in the simulation, and this
suggests that CSC108's jobs do not fit perfectly within the spaces in real
life. This also serves as an estimate for the impact of CSC108's jobs on the
wait times experienced by other projects' jobs on Titan, because an increase of
1.30\% over a three-year period represents an average of sorts for shorter
time periods. This result suggests that CSC108's jobs may impact other jobs on
Titan by increasing their wait times.

If CSC108's jobs fit perfectly within the spaces between other projects' jobs,
then we would expect to observe an increase in throughput, which is defined
here as jobs completed per day, because every CSC108 job that completed would
represent one more job that would not have completed if CSC108's jobs had been
removed. The simulation observed a 14.36\% increase to throughput, but this
number is ambiguous when evaluated on its own. For example, the same number of
Titan core hours can be consumed as one large job or a large number of small
jobs, but in the latter case, throughput may be considerably higher.

Finally, if CSC108's jobs fit perfectly within the spaces between the other
projects' jobs, then we would expect to observe an increase in utilization,
because all utilization by CSC108's jobs would represent utilization of
resources which would otherwise have remained unutilized. The simulation
observed a 1.94\% increase in utilization, but like throughput, this number is
also ambiguous when evaluated on its own. For example, an increase in
utilization could also occur in a scenario where CSC108 was using 100\% of
Titan, causing all other projects to wait.

The results in Table~\ref{tab:rescheduling-results} do imply, however, that
some of the resources consumed by CSC108's jobs in the simulation would have
gone to waste. This is because the product of the time to completion and the
utilization can be used to calculate the total amount of resources consumed
during each simulation run. Without CSC108's jobs included, the remaining jobs
required 1021.2 days of simulation time on Titan to complete, and they consumed
92.36\% of Titan's resources during that time; this is an equivalent amount of
work, when redistributed, to consuming 943.18 days of 100\% of Titan's
resources. By a similar calculation, 973.98 Titan-days of computation were
consumed when CSC108's jobs were included, and this implies that CSC108's jobs
were responsible for 30.8 Titan-days of resource consumption. Because the two
runs' times to completion differed by only 13.3 days, it is impossible for
CSC108 not to have used at least some of the resources which would have gone to
waste.

So, while it is difficult to label exactly which resources would have gone to
waste and when, the simulation results suggest that some non-zero portion of
the resources consumed by CSC108 would have gone to waste. The simulation
results also indicated that CSC108's jobs did not fit perfectly within the
spaces between the other projects' jobs, which suggests that CSC108 ``blocks''
other projects some of the time and that some non-zero portion of the resources
consumed by CSC108 would not have gone to waste.

\jhanote{Need more set-up: why do we choose these three metrics? Define them in text too? Contextualize 1.3\% (collective) versus fluctuation per individual
jobs in queue. Once again, the following claim / statement is wrong: ``''must
have consumed resources which would otherwise have gone to waste.``. Recall
the statement ``must have consumed resources which would otherwise have gone
to waste.`` is a corollary to the hypothesis that ``CSC108 does not impact
Titan''... so you cannot use that to justify the hypothesis.}
\seannote{I killed off the use of language related to a ``hypothesis'' to try
to clean things up, and I used a new data-based argument for why the simulation
suggests that CSC108 partially uses resources which would otherwise have gone
to waste.}

% For tables use
\begin{table}
% table caption is above the table
\caption{Rescheduling study results}
\label{tab:rescheduling-results}       % Give a unique label
% For LaTeX tables use
\begin{tabular}{lrrr}
\hline\noalign{\smallskip}
\phantom{booga}     &   Without CSC108  &   With CSC108 &   Percent change  \\
\hline\noalign{\smallskip}
Time to completion (days)           &   1021.2  &   1034.5  &   1.30        \\
Throughput (jobs per day)           &   1324.93 &   1515.19 &   14.36       \\
Utilization (percent)               &   92.36   &   94.15   &   1.94        \\
\noalign{\smallskip}\hline
\end{tabular}
\end{table}


%%%%%%%%%%%%%%%%%%%%%%%%%%%%%%%%%%%%%%%%%%%%%%%%%%%%%%%%%%%%%%%%%%%%%%%%%%%%%%%%
\subsection{Searching for simple linear relationships}
\label{subsec:simple-linear-relationships}

\jhanote{relationships of what to what? is the result being used as the
subsection title?}
\seannote{I don't know what to call it. I looked across dozens of combinations
of variables with ordinary least squares regression and never found anything
conclusive. I had to give it a title of some kind. Naming is hard.}

% \jhanote{Please be specific about the ecosystem, viz., the QOS for other users
% as measured by their average wait time.} \jhanote{of what??} \jhanote{unclear opening. I think it needs rewrite.} \seannote{I rewrote it. It sounds similar but
% it has more details \ldots ?}

Recall that the goal of CSC108 was to consume resources on Titan which would
otherwise have gone unutilized, while not decreasing the quality of service for
other projects on Titan. The results of Section~\ref{subsec:rescheduling-study}
suggested that CSC108 consumed resources which might have gone to waste, but
that it also may have impacted other projects' quality of service. This section
details further explorations to understand the impact of CSC108 on Titan by
searching for simple linear relationships, especially direct and inverse
relationships, between variables like bin size and indicators like wait times,
throughput, and utilization, by using ordinary least-squares regression.

The data used for this experiment include the same historical trace data used
in Section~\ref{subsec:rescheduling-study}, supplemented with daily availability
data for Titan provided by OLCF for the same years, 2016-2018. The experimental
programs were written in Python using the well-known Matplotlib, NumPy, SciPy,
and scikit-learn libraries via Anaconda. Data were originally provided as text
files but were imported into a SQLite database. We plotted best-fit lines with
ordinary least squares (OLS) linear regression, constructed 95\% confidence
regions around the lines, and used basic measures for goodness-of-fit such as
R-squared.

There were two main ideas used here. The first idea was to look for a
relationship between CSC108's throughput and the throughput of all other jobs
on Titan, and second idea was to look for a relationship between CSC108's
utilization as compared with overall utilization on Titan. Additionally, the
same ideas were repeated to look for impacts due to different sizes of CSC108's
jobs, using the bins defined in Table~\ref{tab:olcf-bins}, because jobs are
assigned priority differently by the scheduler on Titan in part due to the
number of requested nodes and length of requested wall time.

%%%
% THROUGHPUT FIGURES AS SUBFIGURES
%%%
\begin{figure*}
  \subfloat[All\label{fig:throughput-all}]{
    \includegraphics[width=0.4\textwidth]{images/linfit-throughput-all.png}}
  \subfloat[Bin 3\label{fig:throughput-bin3}]{
    \includegraphics[width=0.4\textwidth]{images/linfit-throughput-bin3.png}}
  \vspace{1em}
  \subfloat[Bin 4\label{fig:throughput-bin4}]{
    \includegraphics[width=0.4\textwidth]{images/linfit-throughput-bin4.png}}
  \subfloat[Bin 5\label{fig:throughput-bin5}]{
    \includegraphics[width=0.4\textwidth]{images/linfit-throughput-bin5.png}}
  \caption{This figure demonstrates the relationship between CSC108 backfill
throughput and throughput of other projects on Titan, in terms of jobs
completed per day. Each blue point represents one day. Each red line is an
Ordinary Least Squares (OLS) linear regression with parameters given in
Table~\ref{tab:throughput-params}. Each shaded gray area represents a 95\%
confidence region. Each horizontal dotted black line represents the mean number
of jobs completed on Titan every day by projects other than CSC108, and the
vertical dotted black line represents the mean number of jobs completed every
day by CSC108's use of backfill opportunity.}
\end{figure*}

% For tables use
\begin{table}
% table caption is above the table
\caption{The table contains the parameter values for the Ordinary Least Squares
(OLS) linear regression models regarding throughput. The first column
corresponds to the figure depicting the model, and the second column
corresponds to the OLCF bin number, as defined in Table~\ref{tab:olcf-bins}.
The second and third columns correspond the coefficients $\beta_1$ and
$\beta_0$ in the model $y = \beta_{1}x + \beta_0$.}
\label{tab:throughput-params}       % Give a unique label
% For LaTeX tables use
\begin{tabular}{ccrrr}
\hline\noalign{\smallskip}
Figure & OLCF Bin & Slope $\beta_1$  & Intercept $\beta_0$  &   $\text{R}^2$ \\
\noalign{\smallskip}\hline\noalign{\smallskip}
\ref{fig:throughput-all}    &   All &   0.4106  &   1164.2561   & 0.0040    \\
\ref{fig:throughput-bin3}   &   3   &   0.4419  &   1322.0784   & 0.0005    \\
\ref{fig:throughput-bin4}   &   4   &   1.9819  &   1211.3384   & 0.0027    \\
\ref{fig:throughput-bin5}   &   5   &   0.3072  &   1195.6684   & 0.0018    \\
\noalign{\smallskip}\hline
\end{tabular}
\end{table}
%%%

%%%
% UTILIZATION FIGURES AS SUBFIGURES
%%%
\begin{figure*}
  \subfloat[All\label{fig:utilization-all}]{
    \includegraphics[width=0.4\textwidth]{images/linfit-utilization-by-true-day-all.png}}
  \subfloat[Bin 3\label{fig:utilization-bin3}]{
    \includegraphics[width=0.4\textwidth]{images/linfit-utilization-by-true-day-bin3.png}}
  \vspace{1em}
  \subfloat[Bin 4\label{fig:utilization-bin4}]{
    \includegraphics[width=0.4\textwidth]{images/linfit-utilization-by-true-day-bin4.png}}
  \subfloat[Bin 5\label{fig:utilization-bin5}]{
    \includegraphics[width=0.4\textwidth]{images/linfit-utilization-by-true-day-bin5.png}}
  \caption{This figure demonstrates the relationship between CSC108 backfill
utilization and overall utilization on Titan, as percentages of available
node-hours each day. Each blue point represents one day. Each red line is an
Ordinary Least Squares (OLS) linear regression with parameters given in
Table~\ref{tab:utilization-params}. Each shaded gray area represents a 95\%
confidence regions. Each horizontal dotted black line represents the mean
utilization every day on Titan, and each vertical dotted black line represents
the mean utilization of backfill opportunity every day by CSC108.}
\end{figure*}

% For tables use
\begin{table}
% table caption is above the table
\caption{The table contains the parameter values for the Ordinary Least Squares
(OLS) linear regression models regarding utilization. The first column
corresponds to the figure depicting the model, and the second column
corresponds to the OLCF bin number, as defined in Table~\ref{tab:olcf-bins}.
The second and third columns correspond the coefficients $\beta_1$ and
$\beta_0$ in the model $y = \beta_{1}x + \beta_0$.}
\label{tab:utilization-params}       % Give a unique label
% For LaTeX tables use
\begin{tabular}{ccrrr}
\hline\noalign{\smallskip}
Figure  & OLCF Bin & Slope $\beta_1$  & Intercept $\beta_0$  &  $\text{R}^2$ \\
\noalign{\smallskip}\hline\noalign{\smallskip}
\ref{fig:utilization-all}    &   All &  -0.5258 &   93.3404     &   0.0330  \\
\ref{fig:utilization-bin3}   &   3   &  -1.0977 &   94.0609     &   0.1359  \\
\ref{fig:utilization-bin4}   &   4   &  -1.1472 &   92.7870     &   0.0378  \\
\ref{fig:utilization-bin5}   &   5   &   4.3328 &   87.5839     &   0.1046  \\
\noalign{\smallskip}\hline
\end{tabular}
\end{table}
%%%

The plots shown in Figures~\ref{fig:throughput-all}, \ref{fig:throughput-bin3},
\ref{fig:throughput-bin4}, and \ref{fig:throughput-bin5} visually suggest that
CSC108 has little to no effect on other projects' throughputs, but the numbers
in Table~\ref{tab:throughput-params} show that linear relationships explain
very little of the variability in the data. The $\text{R}^2$ values, which
represent goodness-of-fit on a scale of 0 to 1, are very close to 0, indicating
poor fit.

Similar problems exist for the utilization results, but they raise one very
interesting question. The plots shown in Figures~\ref{fig:utilization-all},
\ref{fig:utilization-bin3}, and \ref{fig:utilization-bin4} are all suggestive
of an inverse relationship, but \ref{fig:utilization-bin5} suggests a direct
relationship, by virtue of its positive slope. The $R^2$ values show that
linear relationships explain very little of the variability in the data,
however, as shown in Table~\ref{tab:utilization-params}, because the
$\text{R}^2$ values are very close to 0, on a scale of 0 to 1. Thus, there are
no simple linear relationships at work here, because the goodness-of-fit values
are consistently close to 0.
\jhanote{Previous paragraph is casual and not precise. Needs fixing.}
\seannote{Agreed. I took out judgmental language and moved the opinions to be
stored in the Summary section for now.}

%%%%%%%%%%%%%%%%%%%%%%%%%%%%%%%%%%%%%%%%%%%%%%%%%%%%%%%%%%%%%%%%%%%%%%%%%%%%%%%%
\subsection{Blocking probability}
\label{subsec:blocking-probability}

The previous subsection analyzed traces for correlations between global set of
all jobs on Titan and CSC108 jobs. Although a useful exercise, the results are
at best inconclusive due to intrinsic noisiness which makes discerning
correlations even harder. In this section, we  try to model the underlying
process with the objective of moving beyond correlations and inferring casual
relationships between CSC108 scheduling events and a measure of impact
(utilization, wait time or throughput).

We begin by defining a ``block'' which is an event that occurs when an job
submitted to the batch queue is stalled in the wait state even though there
might be sufficient resources to run that job. A block typically occurs due to
some other job(s) using resources, and therefore the otherwise eligible job
has been ``blocked'' by an already-running job. A blocking event is
interesting because it indicates interference between jobs and competition for
resources even in the presence of available resources. A systematic analysis
of CSC108 has the potential to determine its impact on non-CSC108 jobs.

The data used for this experiment include the same historical trace data used
in Section~\ref{subsec:rescheduling-study} and the daily availability
data for Titan used in Section~\ref{subsec:simple-linear-relationships}, but
this time are supplemented with live snapshot data of the system queue for
Titan. Snapshots were gathered by sampling live data from the Moab scheduler by
polling with Python scripts launched by cron jobs on a data transfer node.
These scripts recorded XML output from the ``showbf'' and ``showq'' commands
into files, and more cron jobs launched other Python scripts to import these
files' sample data into SQLite. These tables contain data about the exact state
of the queues at given times, including active jobs, blocked jobs, eligible
jobs, recently completed jobs, and system information such as active nodes and
available backfill opportunities. Then, experimental programs were written in
Python using the same libraries and database as the previous sections.

Formally, the definitions for a block and a blocking probability follow. Let
$C_i$ be the abstract resources in use by CSC108 at the $i^{\text{th}}$ sample
point in time, and let $U_i$ be the unused (idle) resources remaining on
Titan. We then define a boolean $B_i$ representing a ``block'' event to occur
if there exists at least one job at the $i^{\text{th}}$ sample point which
requests $(C_i + U_i)$ resources or less (when $C_i$ is non-zero). If there is
no job at $i^{\text{th}}$ which requests more than $(C_i + U_i)$ we say a
blocking event $B_i$  does not occur. Summing $B_i$ over all $i$ gives a count
of the number of blocking events that occur, and dividing that count by the
number of total sample points yields a quantity we call a ``blocking
probability''.

Informally, blocking probability represents the proportion of samples 
\jhanote{sample of what? time? intervals possibly better term?} in which a
block occurred. The idea here is that when blocking probability increases, it
indicates that the system is experiencing greater competition for its
resources. Blocking probability does not predict the probability that a
particular job will be blocked, but rather the probability that a given sample
will contain a block.

To apply this abstract model to a real data set, we have initially defined the
resources in one-dimensional ``spatial'' and ``temporal'' manners, by
considering only jobs' requested numbers of nodes in the former and only jobs'
requested wall times in the latter. An eligible job in the batch queue is said
to be spatially blocked when the job’s number of requested nodes is too large
to fit within the nodes available through backfill opportunity, so that the job
must wait to run. Similarly, an eligible job in the batch queue is said to be
temporally blocked when the job’s requested wall time is too long to fit within
the duration available through backfill opportunity. Similarly, a job is said
to be blocked ``due to CSC108'' if at least one job which was blocked would no
longer be blocked if CSC108's jobs were removed. Thus, a job is only said to be
blocked due to CSC108 if it requests resources with are greater than $U_i$ but
less than $(C_i + U_i)$. Figures
\ref{fig:spatial-blocking-by-month} and \ref{fig:temporal-blocking-by-month}
demonstrate how spatial and temporal blocking probabilities vary from month to
month, and Figure~\ref{fig:spatial-vs-temporal} shows that the two quantities
relate to each other in an intuitive way, namely, that time periods of greater
spatial blocking often correspond to time periods of greater temporal blocking
as well.

%%%
% WAIT TIMES AS SUBFIGURES
\begin{figure*}
  \subfloat[Spatial blocking\label{fig:spatial-blocking-by-month}]{
    \includegraphics[width=0.4\textwidth]{images/barplot-spatial-blocking-by-month.png}}
  \subfloat[Temporal blocking\label{fig:temporal-blocking-by-month}]{
    \includegraphics[width=0.4\textwidth]{images/barplot-temporal-blocking-by-month.png}}
  \caption{These plots depict the spatial and temporal blocking probabilities
by month for samples in which CSC108 was actively utilizing backfill
opportunity. The total height of the bars indicates the blocking probability
for the month, which is the proportion of samples in which at least one
eligible job was blocked. The red region indicates the percentage of samples in
which at least one eligible job would no longer be blocked if CSC108's jobs
were removed.}
\end{figure*}

% For two-column wide figures use
\begin{figure*}
% Use the relevant command to insert your figure file.
% For example, with the graphicx package use
  \includegraphics[width=0.75\textwidth]{images/linfit-spatial-vs-temporal-by-day.png}
% figure caption is below the figure
\caption{This figure demonstrates the relationship between spatial and temporal
blocking probabilities. Each blue point represents one day. The red line is an
Ordinary Least Squares (OLS) linear regression ($y = \beta_{1}x + \beta_0$)
with a slope $\beta_1$ of 0.2503 and an intercept $\beta_0$ of 68.7731. The
shaded gray areas represent 95\% confidence regions. The horizontal dotted
black line represents the mean spatial blocking probability for all points, and
the vertical dotted black line represents the mean temporal blocking
probability for all points. The $\text{R}^2$ value is 0.4410.}
\label{fig:spatial-vs-temporal}
\end{figure*}
%%%

Three indicators of system performance were chosen this time, as well, to
assess the impact of CSC108 on Titan: wait times, throughput, and utilization.
In order to map wait time to a value that can be attributed to a day, wait time
was defined in terms of an average wait time. Average wait time was defined as
the total number of hours spent waiting during a given day, per job that
appeared on that day. For example, a job which was submitted one day but which
did not run until the next day would contribute part of its wait time to the
first day and the rest to the second day, and it would be considered to have
appeared on both days. Throughput was defined as the number of jobs completed
per day, as before. Utilization was also defined as before, as the percentage
of core hours consumed out of the total core hours available.

%%% WAIT TIMES STUFF

Having established the two measures of blocking probability and their
relationship to one another, we followed the same techniques used in
Section~\ref{subsec:simple-linear-relationships} to create best-fit lines with
95\% confidence intervals, to investigate the relationships between blocking
probabilities and wait times experienced by jobs on Titan. Figures
\ref{fig:wait-time-spatial-all} and \ref{fig:wait-time-temporal-all} illustrate
the effects of spatial and temporal blocking probability on wait times, and
Figures \ref{fig:wait-time-spatial-csc108} and
\ref{fig:wait-time-temporal-csc108} show how CSC108's contribution to blocking
impacts wait times. More specifically, in Figures
\ref{fig:wait-time-spatial-csc108} and \ref{fig:wait-time-temporal-csc108}, the
values used for the blocking probabilities correspond to the red regions in
Figures \ref{fig:spatial-blocking-by-month} and
\ref{fig:temporal-blocking-by-month}, which indicate the percentage of samples
in which at least one eligible job would no longer be blocked if CSC108 freed
its resources. The qualitative interpretation for the wait time plots is that,
as competition for resources increases on Titan, average wait times decrease,
but when competition with CSC108 for nodes increases, average wait times
increase. Unfortunately, the goodness-of-fit values are again very poor.

\begin{figure*}
  \subfloat[Spatial blocking\label{fig:wait-time-spatial-all}]{
    \includegraphics[width=0.4\textwidth]{images/linfit-wait-time-vs-spatial-blocking-by-day.png}}
  \subfloat[Temporal blocking\label{fig:wait-time-temporal-all}]{
    \includegraphics[width=0.4\textwidth]{images/linfit-wait-time-vs-temporal-blocking-by-day.png}}
  \vspace{1em}
  \subfloat[Spatial blocking by CSC108\label{fig:wait-time-spatial-csc108}]{
    \includegraphics[width=0.4\textwidth]{images/linfit-wait-time-vs-csc108-spatial.png}}
  \subfloat[Temporal blocking by CSC108\label{fig:wait-time-temporal-csc108}]{
    \includegraphics[width=0.4\textwidth]{images/linfit-wait-time-vs-csc108-temporal.png}}
  \caption{These plots demonstrate the relationships between the average wait
times on Titan and one-dimensional blocking probabilities. Each blue point
represents one day. Each red line is an Ordinary Least Squares (OLS) linear
regression with parameters given in Table~\ref{tab:blocking-wait-time-params}.
Each shaded gray area represents a 95\% confidence region. Each horizontal
dotted black line represents the mean wait times for all points in that plot,
and each vertical dotted black line represents the mean blocking probability
for all points in that plot.}
\end{figure*}

% For tables use
\begin{table}
% table caption is above the table
\caption{The table contains the parameter values for the Ordinary Least Squares
(OLS) linear regression models regarding blocking probabilities and average
wait times. The first column corresponds to the figure depicting the model,
while the second and third columns correspond the coefficients $\beta_1$ and
$\beta_0$ in the model $y = \beta_{1}x + \beta_0$.}
\label{tab:blocking-wait-time-params}       % Give a unique label
% For LaTeX tables use
\begin{tabular}{crrr}
\hline\noalign{\smallskip}
Figure  & Slope $\beta_1$ & Intercept $\beta_0$     & $\text{R}^2$ \\
\noalign{\smallskip}\hline\noalign{\smallskip}
\ref{fig:wait-time-spatial-all}     &   -0.0810 &   11.8610 &   0.0737  \\
\ref{fig:wait-time-temporal-all}    &   -0.0401 &    7.7491 &   0.1265  \\
\ref{fig:wait-time-spatial-csc108}  &    0.0219 &    3.2420 &   0.0509  \\
\ref{fig:wait-time-temporal-csc108} &   -0.0102 &    5.3217 &   0.0147  \\
\noalign{\smallskip}\hline
\end{tabular}
\end{table}

%%% THROUGHPUT STUFF

Figures \ref{fig:throughput-spatial-all}, \ref{fig:throughput-temporal-all},
\ref{fig:throughput-spatial-csc108}, and \ref{fig:throughput-temporal-csc108}
are all in agreement that increasing competition corresponds to increasing
throughput, in units of jobs completed per day. The goodness-of-fit values are
poor, however, as shown in Table~\ref{tab:blocking-throughput-params}, so these qualitative results may only be said to be suggestive.

%%%
\begin{figure*}
  \subfloat[Spatial blocking\label{fig:throughput-spatial-all}]{
    \includegraphics[width=0.4\textwidth]{images/linfit-throughput-vs-spatial-blocking.png}}
  \subfloat[Temporal blocking\label{fig:throughput-temporal-all}]{
    \includegraphics[width=0.4\textwidth]{images/linfit-throughput-vs-temporal-blocking.png}}
  \vspace{1em}
  \subfloat[Spatial blocking by CSC108\label{fig:throughput-spatial-csc108}]{
    \includegraphics[width=0.4\textwidth]{images/linfit-throughput-vs-csc108-spatial.png}}
  \subfloat[Temporal blocking by CSC108\label{fig:throughput-temporal-csc108}]{
    \includegraphics[width=0.4\textwidth]{images/linfit-throughput-vs-csc108-temporal.png}}
  \caption{These plots demonstrate the relationships between throughput on
Titan and one-dimensional blocking probabilities. Each blue point represents
one day. Each red line is an Ordinary Least Squares (OLS) linear regression
with parameters given in Table~\ref{tab:blocking-throughput-params}. Each
shaded gray area represents a 95\% confidence region. Each horizontal dotted
black line represents the mean wait times for all points in that plot, and each
vertical dotted black line represents the mean blocking probability for all
points in that plot.}
\end{figure*}

% For tables use
\begin{table}
% table caption is above the table
\caption{The table contains the parameter values for the Ordinary Least Squares
(OLS) linear regression models regarding blocking probabilities and throughput.
The first column corresponds to the figure depicting the model, while the
second and third columns correspond the coefficients $\beta_1$ and $\beta_0$ in
the model $y = \beta_{1}x + \beta_0$.}
\label{tab:blocking-throughput-params}       % Give a unique label
% For LaTeX tables use
\begin{tabular}{crrr}
\hline\noalign{\smallskip}
Figure  & Slope $\beta_1$ & Intercept $\beta_0$     & $\text{R}^2$ \\
\noalign{\smallskip}\hline\noalign{\smallskip}
\ref{fig:throughput-spatial-all}     &  16.2402 &   252.3652    &   0.0122  \\
\ref{fig:throughput-temporal-all}    &   1.7196 &  1544.9669    &   0.0010  \\
\ref{fig:throughput-spatial-csc108}  &  13.4683 &   730.0687    &   0.0790  \\
\ref{fig:throughput-temporal-csc108} &  10.0245 &  1134.0212    &   0.0587  \\
\noalign{\smallskip}\hline
\end{tabular}
\end{table}
%%%

%%% UTILIZATION STUFF

Finally, we searched for simple linear relationships between the different
blocking probabilities and overall utilization on Titan. Figures
\ref{fig:utilization-spatial-all}, \ref{fig:utilization-temporal-all},
\ref{fig:utilization-spatial-csc108}, and \ref{fig:utilization-temporal-csc108}
do not ``agree'' like the throughput plots did, but three plots suggest an
interpretation in which increasing competition, indicated by increasing
blocking probability, corresponds to decreased utilization. The fourth plot,
which indicates competition with CSC108, relates increased competition to
increased utilization. Once again, the goodness-of-fit values are poor, as
shown in Table~\ref{tab:blocking-utilization-params}.

%%%
\begin{figure*}
  \subfloat[Spatial blocking\label{fig:utilization-spatial-all}]{
    \includegraphics[width=0.4\textwidth]{images/linfit-utilization-vs-spatial-blocking.png}}
  \subfloat[Temporal blocking\label{fig:utilization-temporal-all}]{
    \includegraphics[width=0.4\textwidth]{images/linfit-utilization-vs-temporal-blocking.png}}
  \vspace{1em}
  \subfloat[Spatial blocking by CSC108\label{fig:utilization-spatial-csc108}]{
    \includegraphics[width=0.4\textwidth]{images/linfit-utilization-vs-csc108-spatial.png}}
  \subfloat[Temporal blocking by CSC108\label{fig:utilization-temporal-csc108}]{
    \includegraphics[width=0.4\textwidth]{images/linfit-utilization-vs-csc108-temporal.png}}
  \caption{These plots demonstrate the relationships between utilization on
Titan and one-dimensional blocking probabilities. Each blue point represents
one day. Each red line is an Ordinary Least Squares (OLS) linear regression
with parameters given in Table~\ref{tab:blocking-utilization-params}. Each
shaded gray area represents a 95\% confidence region. Each horizontal dotted
black line represents the mean wait times for all points in that plot, and each
vertical dotted black line represents the mean blocking probability for all
points in that plot.}
\end{figure*}

% For tables use
\begin{table}
% table caption is above the table
\caption{The table contains the parameter values for the Ordinary Least Squares
(OLS) linear regression models regarding blocking probabilities and utilization.
The first column corresponds to the figure depicting the model, while the
second and third columns correspond the coefficients $\beta_1$ and $\beta_0$ in
the model $y = \beta_{1}x + \beta_0$.}
\label{tab:blocking-utilization-params}       % Give a unique label
% For LaTeX tables use
\begin{tabular}{crrr}
\hline\noalign{\smallskip}
Figure  & Slope $\beta_1$ & Intercept $\beta_0$     & $\text{R}^2$ \\
\noalign{\smallskip}\hline\noalign{\smallskip}
\ref{fig:utilization-spatial-all}       &   -0.3766 &   123.8332 & 0.1543 \\
\ref{fig:utilization-temporal-all}    &     -0.1654 &   103.1603 & 0.2084 \\
\ref{fig:utilization-spatial-csc108}  &      0.0617 &    86.5830 & 0.0391 \\
\ref{fig:utilization-temporal-csc108} &     -0.0518 &    93.6845 & 0.0370 \\
\noalign{\smallskip}\hline
\end{tabular}
\end{table}
%%%

Thus, the use of blocking probability provided additional insight regarding the
impact of CSC108 on Titan, but just like in
Section~\ref{subsec:simple-linear-relationships}, the best-fit lines all
displayed very poor goodness-of-fit, rendering the interpretations somewhat
weak.

%-  vim:set syntax=tex:


% ---------------------------------------------------------------------------
% V - Workload Management for HPC Beyond ATLAS and OLCF
% ---------------------------------------------------------------------------

\section{HPC Workload Management beyond ATLAS and OLCF}
\label{sec:beyond-atlas-and-olcf}
%-  LaTeX source file

%-  section5.tex ~~
%
%   This contains what was originally the seventh section of the Google Docs
%   draft of the paper.
%
%                                                   ~~ last updated 20 Sep 2019

The PanDA WMS was originally developed to meet the needs of ATLAS, where it has
been handling production workloads since 2005. PanDA's design was not only
informed by the needs of ATLAS but also optimized to meet them, and similarly,
ATLAS workloads were tailored to run through PanDA. Despite this long history
of co-evolution, nothing about PanDA restricts its use to running physics
simulations on Titan or the WLCG. This section describes some of the
applications of PanDA for managing workloads in other scientific disciplines,
using Titan and other HPC resources.

%%%%%%%%%%%%%%%%%%%%%%%%%%%%%%%%%%%%%%%%%%%%%%%%%%%%%%%%%%%%%%%%%%%%%%%%%%%%%%%%
\subsection{Common Themes for PanDA-Based Workflows}
\label{subsec:common-themes}


Traditionally, the basic pattern for computing in support of physics
experiments is to process input files to produce output files. In this paper,
we have referred to this processing in terms of discrete jobs. These jobs often
require that further input parameters be provided to experiment-specific
software in the form of command-line arguments or additional configuration
files. PanDA can manage a variety of different workloads because it can handle
many different ways of specifying jobs' input and output. For example,
job-specific configuration files can be defined in a PanDA job description
simply as additional input files, and log files created by experiment-specific
software can be defined as additional output files to be saved in the same
tarball that the PanDA pilot creates to save its own log files.


PanDA manages all this processing according to ``transformation scripts'' which
contain full descriptions that provide all necessary information like
configuration and execution parameters for running jobs and saving output. A
PanDA job definition only defines the ``payload'', which is the launching
command for the transformation script. Experiments like ATLAS need to define
complete sets of transformation scripts that cover all possible software usage,
but in many cases, a single transformation script suffices.

PanDA is an attractive solution for new experiments for many reasons. PanDA
provides high-level features for automating the job handling, monitoring, and
logging without removing the ability to control and customize stages in a
workflow. It streamlines the usage of computing resources, especially when
federating resources from multiple facilities, and it insulates users from
details like policies and schedulers that are specific to a local resource.
PanDA provides a diverse set of plugins to support staging data in and out from
remote storages and using different protocols. Another important feature of
PanDA is that is has the ability to integrate closely with the systems at OLCF,
and this makes it attractive both to existing and aspiring OLCF users.


Currently, there are several instances of PanDA Server in use by different
experiments and groups, but in this paper, we have considered three main
instances. The original instance is installed at CERN, and it is used
exclusively for the ATLAS experiment. Another instance is installed at OLCF, as
outlined in Section~\ref{subsec:panda_instance}, and it is dedicated to
supporting projects on Titan, subject to OLCF policies. Finally, the instance
on Amazon's EC2 cloud infrastructure provides access to multiple independent
experiments from different disciplines, and it has the least restrictive
security and usage policies.

Using the OpenShift instance of PanDA outlined in
Section~\ref{subsec:panda_instance}, we created demonstrations for
orchestrating workflows from various scientific fields on OLCF resources. To
isolate the workflows of different groups and experiments, dedicated queues
were defined on each instance of PanDA Server. Separating different groups'
workflows by using different queues provides advantages in customizing
environment variables, system settings, and workflows for different user
groups, and it simplifies job monitoring via the web-based interface. In
collaboration with representatives from other scientific groups, we implemented
transformation scripts as command-line tools that can be addressed by name. A
client tool provided to the users allows them to submit jobs to PanDA Server
with authentication based on grid certificates. Each group's representatives
were authorized to run the pilot launcher daemon with a configurable amount of
concurrency. In this way, the pilots ran and interacted with PBS using the user
and project privileges of the research group, as opposed to running under the
BigPanDA team's account, CSC108.


%%%%%%%%%%%%%%%%%%%%%%%%%%%%%%%%%%%%%%%%%%%%%%%%%%%%%%%%%%%%%%%%%%%%%%%%%%%%%%%%
\subsection{Managing Non-ATLAS Workloads on Titan}
\label{subsec:other-workloads}

The BigPanDA team collaborated with experts from research groups in other
scientific disciplines in order to demonstrate PanDA's abilities to manage
different kinds of workloads on Titan at OLCF. Four representative experiments
are briefly described here, the details for which are summarized in
Table~\ref{tab:beyondhep}.

In the field of genomics, the BigPanDA team collaborated with the Center for
Bioenergy Innovation at ORNL to construct a PanDA-based workflow for genomics
research focused on epistasis. Epistasis is the phenomenon in which the effect
of one gene is dependent on the presence of one or more ``modifier genes''. The
payload used in this workflow is GBOOST \cite{GBOOST}, which is a GPU-based
implementation of the Boolean Operation-based Screening and Testing (BOOST)
software for detecting gene-gene interactions in genome-wide association
studies (GWAS). Each job used only 2 nodes for 30 minutes of wall time to
process 100 MB of input data and produce 300 MB of output data. These jobs were
ideal candidates for using PanDA on Titan because they were small, short, and
GPU-based, which meant that they can take advantage of Titan's hybrid
architecture using backfill opportunities.

In molecular dynamics, the BigPanDA team collaborated with the Chemistry and
Biochemistry Department at the University of Texas at Arlington to construct a
PanDA-based workflow for simulating enzyme catalysis, conformational change,
and ligand binding and release. The payload used in this workflow is CHARMM
(Chemistry at HARvard Macromolecular Mechanics) \cite{Brooks2009CHARMM}, which
is a molecular simulation program capable of using hybrid MPI, OpenMP, and GPU
computing. These jobs ranged in size, but each typically used 124 nodes for
30-90 minutes of wall time. They required just 10 KB of input data files but
produced 2-6 GB of output data. These workloads are interesting because they
can scale to many hundreds of nodes while not requiring large amounts of wall
time, which means they can potentially take great advantage of backfill
opportunities, too.

In particle physics, the BigPanDA team collaborated with the IceCube experiment
\cite{Halzen:2010yj} to construct a PanDA-based workflow for neutrino event
simulation. IceCube is a particle detector at the South Pole that records the
interactions of a nearly massless subatomic particle called the neutrino. The
payload used in this workflow is NUGEN, which is an experiment-specific
software based on ANIS \cite{Gazizov:2004va} that uses GPU computing to
generate samples of atmospheric neutrinos for analysis. These jobs only
required one node at a time for 120 minutes in order to process 500 KB of input
data, but there were many, many jobs to perform as part of their scientific
campaign, and each job produced a volume of output data that could be anywhere
from 10 KB all the way to 4 GB. These workloads were interesting because they
required the use of Singularity containers \cite{} as well as remote stage-in
and stage-out of the data from GridFTP \cite{Allcock:2005:GSG:1105760.1105819}
storage with GSI authentication.

Finally, in neutron science, the BigPanDA team collaborated with the Neutron
Electric Dipole Moment (nEDM) experiment \cite{0954-3899-36-10-104002} to
construct a PanDA-based workflow for processing experimental and observational
data from the Fundamental Neutron Physics Beamline at the Spallation Neutron
Source at ORNL. The goal of the nEDM experiment is to improve the precision of
the measurements of the properties of the neutron in order to search for
violations of fundamental symmetries and to make critical tests of the validity
of the Standard Model of electroweak interactions. The payload used in this
workflow is GEANT \cite{}, which is the same software used by ATLAS to simulate
the passage of particles through matter. These jobs required 200 nodes at a
time, but only for 20 minutes of wall time, in order to process 120 MB of input
data and produce 20 MB of output data. These workloads are particularly
interesting because it is an experimental/observational facility (EOF) which is
local to OLCF resources.


%%%%%%%%%%%%%%%%%%%%%%%%%%%%%%%%%%%%%%%%%%%%%%%%%%%%%%%%%%%%%%%%%%%%%%%%%%%%%%%%
\subsection{Managing Workloads for HPC on Non-OLCF Resources}
\label{subsec:other-hpc-resources}


The BigPanDA team also collaborated on workload management for HPC resources
beyond just Titan at OLCF. Three of these experiments are briefly described
here to represent some of these collaborations, the details for which are
summarized in Table~\ref{tab:beyondhep}.

In the field of astronomy, the BigPanDA team collaborated with the Large
Synoptic Survey Telescope (LSST) project to construct a PanDA-based workflow
for Monte Carlo simulation of image data. Beginning in 2022, LSST will conduct
a 10-year survey of the sky that is expected to deliver 200 PB of data in order
to address some of the most pressing questions about the structure and
evolution of the universe and the objects in it. The payload used in this
workflow is PhoSim \cite{}, a set of fast photon Monte Carlo codes used to
calculate the physics of the atmosphere and a telescope and camera in order to
simulate realistic astronomical images. Each job in the demonstration workflow
used only 2 nodes but required 600 minutes of wall time to process 700 MB of
input data and produce 70 MB of output data. For running LSST simulations with
the PanDA WMS, we have established a distributed testbed infrastructure that
employs the resources of several sites on GridPP \cite{0954-3899-32-1-N01} and
Open Science Grid (OSG) \cite{1742-6596-78-1-012057} as well as the Titan
supercomputer at ORNL. In order to submit jobs to these sites we have used a
PanDA server instance deployed on the Amazon AWS Cloud.

In the field of nuclear physics, and as a part of a SciDAC-4 funded project,
the BigPanDA team collaborated with the Lattice QCD (LQCD)
\cite{Babich:2010:PQL:1884643.1884695} experiment to construct a PanDA-based
workflow to meet the needs of the SciDAC-4 LQCD computational program. LQCD is
a well-established, non-perturbative approach to solving the quantum
chromodynamics theory of quarks and gluons. Current LQCD payloads can be
characterized as massively parallel, occupying thousands of nodes on
leadership-class supercomputers. LQCD payloads have been successfully tested on
Titan as well as on other sites. Production campaigns were executed on the
Institutional Cluster at Brookhaven National Laboratory (BNL) through a
dedicated instance of Harvester \cite{} installed on the site's front node.
Between April and June 2018, PanDA managed workloads that processed 13 TB of
input data and produced 176 GB of output data. LQCD jobs used around 15,000 GPU
hours with an average job duration of approximately 12 hours.

Finally, in the field of neuroscience, the BigPanDA team collaborated with the
Blue Brain Project (BBP) \cite{Markram} of the Ecole Polytechnique Federal de
Lausanne (EPFL) in Switzerland. This proof-of-concept project was aimed at
demonstrating the efficient application of the BigPanDA system to support the
complex scientific workflow of the BBP, which relies on using a mix of
desktops, clusters, and supercomputers to reconstruct and simulate accurate
models of brain tissue. In the first phase of this joint project, we supported
the execution of BBP software on a variety of distributed computing systems
powered by BigPanDA. The targeted systems for demonstration included
cloud-based resources on Amazon Cloud, two clusters using Intel x86 and Nvidia
GPUs located in Geneva and Lugano, one IBM Blue Gene/Q \cite{citeulike:472727}
supercomputer which was also located in Lugano, and of course, Titan at OLCF.


% For tables use
\begin{table}
% table caption is above the table
\caption{Summary of job parameters for non-ATLAS workloads deployed to Titan at
OLCF.}
\label{tab:beyondhep}       % Give a unique label
% For LaTeX tables use
\begin{tabular}{llrrrrr}
\hline\noalign{\smallskip}
% ORIGINAL COLUMN TITLES:
%Experiment & Payload/SW & Number of jobs per campaign & Number of nodes per
%job & Walltime (min) & Input data size per job & Output data size per job \\
Experiment & Payload & Jobs & Nodes & Walltime & Input data & Output data \\
\noalign{\smallskip}\hline\noalign{\smallskip}
Genomics           & GBOOST & 10    & 2    & 30 min    & 100 MB & 300 MB \\
IceCube            & NuGen  & 4500K & 1    & 120 min   & 500 KB & 10KB - 4GB \\
LSST/DESC          & PhoSim & 20    & 2    & 600 min   & 700 MB & 70 MB \\
LQCD               & QDP++  & 10    & 8000 & 700 min   & 40 GB  & 150 MB \\
Molecular Dynamics & CHARMM & 10    & 124  & 30-90 min & 10 KB  & 2-6 GB \\
nEDM               & GEANT  & 10    & 200  & 20 min    & 120 MB & 20 MB \\
\noalign{\smallskip}\hline
\end{tabular}
\end{table}


%-  vim:set syntax=tex:


%%%

% ---------------------------------------------------------------------------
% VI - Summary
% ---------------------------------------------------------------------------

\section{Summary}
\label{sec:summary}
%-  LaTeX source file

%-  section6.tex ~~
%                                                   ~~ last updated 22 Sep 2019

% Section 4 summary

Recall that the goal of CSC108 has been to consume idle resources on Titan
which would have otherwise gone to waste, while making a good-faith effort not
to disturb the rest of Titan's ecosystem.

The results of the rescheduling study in
Section~\ref{subsec:rescheduling-study} suggested that CSC108 successfully
accomplishes its goal of consuming idle resources which would otherwise have
gone to waste, and also they suggested that CSC108 may have an impact on Titan.
The results of searching for simple linear relationships in
Section~\ref{subsec:simple-linear-relationships} between indicators like
throughput and utilization provided additional insight, but the interpretations
were weak statistically because of the poor goodness-of-fit values. Finally,
blocking probability was used as a means to identify and analyze times of great
competition for resources, but again the interpretations were weak due to poor
goodness-of-fit.


The plots shown in Figures~\ref{fig:utilization-all},
\ref{fig:utilization-bin3}, and \ref{fig:utilization-bin4} are all suggestive
of an inverse relationship, but \ref{fig:utilization-bin5} suggests a direct
relationship, by virtue of its positive slope. This raises the question, what
has caused the sign change? It can be tempting to assign blame and credit in
such a case, such as to say that CSC108's consumption in bin 5 causes an
increase in overall utilization, while its consumption in other bins decreases
overall utilization. Here, however, we are only looking for relationships in
the data, and the goodness-of-fit values are consistently poor. Thus, there
were no simple linear relationships, and this suggests that CSC108 actually did
accomplish its goal of not decreasing the quality of service for other projects
on Titan.

These overall results underscore the difficulty of the main problem of this
section, which was to identify and analyze the impact of the CSC108 project on
Titan. The original hypothesis stated that CSC108 has no effect on Titan. We
can see that it has had an impact on Titan in the form of consuming hundreds of
millions of core hours. Results suggest that CSC108 negatively impacts Titan by
increasing wait times, that CSC108 positively impacts Titan by increasing
throughput, and that CSC108 positively impacts Titan by increasing utilization.
Interestingly, the inability to find simple relationships by using blocking
probability suggests that users' judging system performance by monitoring the
batch queue is similarly incapable. In any case, the difficulty in confirming
any impact may simply provide evidence that the CSC108 project has impacted
Titan minimally, at least with respect to the indicators used.

Finally, we note here that the phenomenon of ``draining'' on Titan may play a
role in some of the counterintuitive results, such as those depicted in Figures
\ref{fig:utilization-all}, \ref{fig:utilization-bin3}, and
\ref{fig:utilization-bin4}. Draining is technically a node state in the Moab
scheduler, but it is used here colloquially to refer to the process by which
the scheduler allows busy nodes to finish executing workload before keeping
them idle, in order to prepare for a capability class (bin 1) job. During this
process, utilization would normally decrease monotonically, but because Titan
has enabled backfill scheduling, these idle resources may actually be used for
small, short jobs, provided that they will complete before those nodes will be
needed for the large job. Because CSC108's consumption increases during times
of increased backfill opportunity, the data will show that times of decreased
utilization are correlated with increased utilization, unless CSC108 is able to
consume all of the backfill opportunity. Even if the project had an infinite
supply of new workloads to submit to Titan, it is limited to 20 concurrently
executing jobs. Thus, CSC108 will appear to increase in utilization at the
same time that all other utilization decreases, even though CSC108 is not
actually displacing other projects' workloads. Future work will examine
draining in greater detail, to determine if these times have an identifiable
signature so that comparisons can be made, in much the same manner as followed
in the blocking probability study. This will allow us to understand whether
the results of Figure~\ref{fig:utilization-bin5} imply that CSC108 should
restrict its individual jobs to use only bin 5, for example.

% Section 5 summary

The overview of the successfully implemented demonstrations of diverse
workflow implementations via PanDA shows that the PanDA model can handle the
challenges of the different experiments and user groups and also provide the
possibility for extensions beyond the set of core components. Pre-production
utilization of PanDA is now under investigation with BlueBrain, IceCube, LSST,
and nEDM experiments. LQCD uses PanDA for production.

%-  vim:set syntax=tex:


%%%

%% For one-column wide figures use
%\begin{figure}
%% Use the relevant command to insert your figure file.
%% For example, with the graphicx package use
%  \includegraphics{images/example.eps}
%% figure caption is below the figure
%\caption{Please write your figure caption here}
%\label{fig:1}       % Give a unique label
%\end{figure}
%%
%% For two-column wide figures use
%\begin{figure*}
%% Use the relevant command to insert your figure file.
%% For example, with the graphicx package use
%  \includegraphics[width=0.75\textwidth]{images/example.eps}
%% figure caption is below the figure
%\caption{Please write your figure caption here}
%\label{fig:2}       % Give a unique label
%\end{figure*}
%%
%

\begin{acknowledgements}
    This work is funded by Award number DESC0016280 from the Office of Advanced
    Scientific Computing Research within the Department of Energy. This
    research used resources of the Oak Ridge Leadership Computing Facility at
    the Oak Ridge National Laboratory, which is supported by the Office of
    Science of the U.S. Department of Energy under Contract
    No.\ DE-AC05-00OR22725.
\end{acknowledgements}


%-  NOTE: I (Sean) chose the bibliography style arbitrarily.

%\bibliographystyle{spbasic}      % basic style, author-year citations
%\bibliographystyle{spmpsci}      % mathematics and physical sciences
\bibliographystyle{spphys}       % APS-like style for physics
\bibliography{bibliography,radical_publications}


%-  That's all, folks!

\end{document}

%-  vim:set syntax=tex:
