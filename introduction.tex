%-  LaTeX source file

%-  introduction.tex ~~
%                                                   ~~ last updated 24 Sep 2018

Traditionally, the ATLAS experiment at LHC has utilized distributed resources
as provided by the WLCG to support data distribution and enable the simulation
of events.  For example, the ATLAS experiment uses a geographically distributed
grid of approximately 200,000 cores continuously (250,000 cores at peak), (over
1,000 million core-hours per year) to process, simulate, and analyze its data
(today's total data volume of ATLAS is more than 300 PB). After the early
success in discovering a new particle consistent with the long awaited Higgs
boson, ATLAS is starting the precision measurements necessary for further
discoveries that will become possible by much higher LHC collision energy and
rates from Run2. The need for simulation and analysis will overwhelm the
expected capacity of WLCG computing facilities unless the range and precision
of physics studies will be curtailed.

Over the past few years, the ATLAS experiment has been investigating the
implications of using high-performance computers -- such as those found at Oak
Ridge leadership class facility (ORNL). This steady transition is a consequence
of application requirements (e.g., greater than expected data production),
technology trends and software complexity.

Our approach to the exascale involve the BigPanDA workload management system
which is responsible for coordination of tasks, orchestration of resources and
job submission and management. Historically, BigPanDA was used to for workload
management across multiple distributed resources on the WLCG. We describe the
changes to the BigPanDA software system needed to enable BigPanDA to utilize
Titan. We will then describe how architectural, algorithmic and software
changes have also been addressed by ATLAS computing.

We quantify the impact of this sustained and steady uptake of supercomputers
via BigPanDA: For the latest 18 month period for which data is available, Big
Panda has enabled the utilization of $\sim$400 Million Titan core
hours (primarily via Backfill mechanisms 275M, but also through regular ``front
end'' submission as part of the ALCC project 125M). This non-trivial amount of
400 million Titan core hours has resulted in 920 million events being analysed.
Approximately 3-5\% of all of ATLAS compute resources now provided by Titan;
other DOE supercomputers provide non-trivial compute allocations. In spite of
these impressive numbers, there is a need to further improve the uptake and
utilization of supercomputing resources to improve the ATLAS prospects for Run
3. 

In spite of these impressive numbers, there is a need to further improve the
uptake and utilization of supercomputing resources to improve the ATLAS
prospects for Run 3. The aim of this paper to (i) \ldots (ii) \ldots (iiii)
\ldots (iv) We will outline how we have steadily made the ATLAS project ready
for the exascale era \ldots 

%-  vim:set syntax=tex:
