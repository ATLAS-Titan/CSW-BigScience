%-  LaTeX source file

%-  section6.tex ~~
%                                                   ~~ last updated 22 Sep 2019

% Section 4 summary

The goal of the project using allocation CSC108 has been to consume idle
resources on Titan which would otherwise have likely gone remain unutilized,
while not having a discernible impact on the quality of service of other jobs
on Titan.

The results of the rescheduling study in
Section~\ref{subsec:rescheduling-study} suggested that CSC108 successfully
accomplishes its goal of consuming idle resources which would otherwise have
gone to waste, and also they suggested that CSC108 may have an impact on Titan.
The results of searching for simple linear relationships in
Section~\ref{subsec:simple-linear-relationships} between indicators like
throughput and utilization provided additional insight, but the interpretations
were weak statistically because of the poor goodness-of-fit values. Finally,
blocking probability was used as a means to identify and analyze times of great
competition for resources, but again the interpretations were weak due to poor
goodness-of-fit.

The plots shown in Figures~\ref{fig:utilization-all},
\ref{fig:utilization-bin3}, and \ref{fig:utilization-bin4} are all suggestive
of an inverse relationship, but \ref{fig:utilization-bin5} suggests a direct
relationship, by virtue of its positive slope. This raises the question, what
has caused the sign change? It can be tempting to assign blame and credit in
such a case, such as to say that CSC108's consumption in bin 5 causes an
increase in overall utilization, while its consumption in other bins decreases
overall utilization. Here, however, we are only looking for relationships in
the data, and the goodness-of-fit values are consistently poor. Thus, there
were no simple linear relationships, and this suggests that CSC108 actually
did accomplish its goal of not decreasing the quality of service for other
projects on Titan.

These overall results underscore the difficulty of the main problem of this
section, which was to identify and analyze the impact of the CSC108 project on
Titan. The original hypothesis stated that CSC108 has no effect on Titan. We
can see that it has had an impact on Titan in the form of consuming hundreds
of millions of core hours. Results suggest that CSC108 negatively impacts
Titan by increasing wait times, that CSC108 positively impacts Titan by
increasing throughput, and that CSC108 positively impacts Titan by increasing
utilization. Interestingly, the inability to find simple relationships by
using blocking probability suggests that users' judging system performance by
monitoring the batch queue is similarly incapable. In any case, the difficulty
in confirming any impact may simply provide evidence that the CSC108 project
has impacted Titan minimally, at least with respect to the indicators used.

Finally, we note here that the phenomenon of ``draining'' on Titan may play a
role in some of the counterintuitive results, such as those depicted in
Figures \ref{fig:utilization-all}, \ref{fig:utilization-bin3}, and
\ref{fig:utilization-bin4}. Draining is technically a node state in the Moab
scheduler, but it is used here colloquially to refer to the process by which
the scheduler allows busy nodes to finish executing workload before keeping
them idle, in order to prepare for a capability class (bin 1) job. During this
process, utilization would normally decrease monotonically, but because Titan
has enabled backfill scheduling, these idle resources may actually be used for
small, short jobs, provided that they will complete before those nodes will be
needed for the large job. Because CSC108's consumption increases during times
of increased backfill opportunity, the data will show that times of decreased
utilization are correlated with increased utilization, unless CSC108 is able
to consume all of the backfill opportunity. Even if the project had an
infinite supply of new workloads to submit to Titan, it is limited to 20
concurrently executing jobs. Thus, CSC108 will appear to increase in
utilization at the same time that all other utilization decreases, even though
CSC108 is not actually displacing other projects' workloads. Future work will
examine draining in greater detail, to determine if these times have an
identifiable signature so that comparisons can be made, in much the same
manner as followed in the blocking probability study. This will allow us to
understand whether the results of Figure~\ref{fig:utilization-bin5} imply that
CSC108 should restrict its individual jobs to use only bin 5, for example.

% Section 5 summary

The overview of the successfully implemented demonstrations of diverse
workflow implementations via PanDA shows that the PanDA model can handle the
challenges of the different experiments and user groups and also provide the
possibility for extensions beyond the set of core components. Pre-production
utilization of PanDA is now under investigation with BlueBrain, IceCube, LSST,
and nEDM experiments. LQCD uses PanDA for production.

%-  vim:set syntax=tex:
